
\section{Ultrametric spaces}
\noindent Standard definitions and facts about ultrametric spaces.

\begin{definition}
	(\textbf{Ultrametric space}) Let $(M, \rho)$ be a metric space. If $\rho$ satistifies the ultrametric inequality
	\[\rho(x,z) \leq max{(\rho(x,y), \rho(y,z))}, \forall x,y,z \in M\] 
	then (M, $\rho$) is an ultrametric space.
\end{definition}

\begin{definition}
	(\textbf{Strong Triangle Inequality}) Let $(V, N)$ be a normed vector space. Then $N$ satisfies the strong trianlge inequality if
	\[N(x + y) \leq max(N(x), N(y)), \forall x,y \in V \]
\end{definition}

\begin{proposition}
	Let $(V,N)$ be a normed vector space and suppose $N$ satisfies the strong triangle inequality. Then the metric space, $(V,\rho_N)$, where $\rho_N$ is the metric induced by $N$, is an ultrametric space.
\end{proposition}


\section{Ultrametric topology}
\noindent Let $(M, \rho)$ be a compact ultrametric space and let $B_r(a)=\{x \in M \mid \rho(x,a) < r\}$ denote the open ball of radius $r$, centred at $a$ for some $r \in \mathbb{R}^+$ and $a \in (M,\rho)$. 

\begin{proposition}
	Let $B_r(a)$ be a ball in an ultrametric space $(M,\rho)$. If $b \in B_r(a)$, then $B_r(a) = B_r(b)$. That is, in an ultrametric space, every point in a ball is at its centre.   
\end{proposition}

\begin{proposition}
	Let $B_r(a)$ be a ball in an ultrametric space $(M,\rho)$. Then the diameter of $B$, $d=diam(B)=\sup_{x,y \in B}{\rho(x,y)}$, is less than or equal to the radius of $B$.    
\end{proposition}

\begin{proposition}
	If $(M, \rho)$ is an ultrametric space and $B_{r_1}(x_0)$ and $B_{r_2}(y_0)$ are balls in $(M, \rho)$, then either $B_{r_1}(x_0) \cap B_{r_2}(y_0) = \emptyset$, $B_{r_1}(x_0) \subseteq B_{r_2}(y_0)$, or $B_{r_2}(x_0) \subseteq B_{r_1}(x_0)$. That is, in an ultrametric space, all balls either overlap or are disjoint.
\end{proposition}

\begin{proposition}
	All triangles are isosceles in an ultrametric space.
\end{proposition}

\noindent Fix an ultrametric space $(M, \rho)$ and give $M$ the metric topology. That is let a basis for the topology be given by the set of open balls, $ \{B_r(a) \mid a \in M, r \in  \mathbb{R}^+\} $. 

\begin{proposition}
	With the notation above, if $S$ is open in $(M, \rho)$, then $S=\cup_i B_{r_i}(x_i)$, where $B_{r_i}(x_i) \cap B_{r_j}(x_j) = \emptyset$ if $i \neq j$.
\end{proposition}

\begin{proposition}
	Let $B_r(a$) be a ball in an ultrametric space $(M,\rho)$. Then $B$ is also a closed set in $(M,\rho)$. In particular, if $(M,\rho)$ is compact, then so is $B$.    
\end{proposition}

\begin{proposition}
	If $(M, \rho)$ is an ultrametric space and $B_{r_1}(x_0)$ and $B_{r_2}(y_0)$ are balls in $(M, \rho)$ then $B_{r_1}(x_0) \setminus B_{r_2}(y_0) = \cup_{i=1}^n B_{r_i}(x_i)$.
\end{proposition}

\begin{proof}
	Let $S = B_{r_1}(x_0) \setminus B_{r_2}(y_0)$. It suffices to show that $S$ is open, since open sets are always disjoint unions of balls. Note that $S = M  \setminus B_{r_2}(y_0) \cup \{x \in M \mid \rho(x, x_0) > r_1\}$. Since the complement of $\{x \in M \mid \rho(x, x_0) > r_1\}$ is  $B_{r_1}(x_0)$ and balls are clopen in an ultrametric space, $B_{r_2}(y_0) \cup \{x \in M \mid \rho(x, x_0) > r_1\}$ is the union of two closed sets, hence closed. Then $S$ is the complement of a closed set, so $S$ must be open. 
\end{proof}



\section{P-adic spaces}
\begin{definition}
	(\textbf{p-adic valuation})
\end{definition}

\begin{definition}
	(\textbf{p-adic absolute value})
\end{definition}

\begin{proposition}
	The p-adic absolute value is multiplicative.
\end{proposition}

\begin{construction}
	$(\mathbb{Z}_p, \mid\cdot\mid_p)$	
\end{construction}	

\begin{corollary}
	$(\mathbb{Z}_p, \mid\cdot\mid_p)$ is a compact ultrametric space.
\end{corollary}

\begin{proposition}
	If $B \subseteq (\mathbb{Z}_p, \mid\cdot\mid_p)$ is a ball, then $B = \{a + p^km\}$, for some $k \in \mathbb{N}, a \in \{0,...,p-1\}$ and $m \in \mathbb{Z}_p$. That is, balls in $(\mathbb{Z}_p, \mid\cdot\mid_p)$ are cosets of the cyclic subgroup generated by $p^k$ for some $k$. 
\end{proposition}

\begin{proposition}
	$(\mathbb{Z}_p \times \mathbb{Z}_p, \mid \cdot \mid_{p,\infty})$ is a compact ultrametric space. 
	\end{proposition}
	
\begin{proof}
	For compactness - by Tychnoff's theorem, it suffices to show the topology induced by the metric coincides with the product topology. The ultrametric property is checked directly.
\end{proof}

\begin{proposition}
	If $B \subseteq (\mathbb{Z}_p \times \mathbb{Z}_p, \mid\cdot\mid_{p,\infty})$ is a ball, then $B = \{a + p^km\} \times \{b + p^lm\}$, for some $k,l \in \mathbb{N}, a,b \in \{0,...,p-1\}$ and $m \in \mathbb{Z}_p$. That is, balls in $(\mathbb{Z}_p \times \mathbb{Z}_p  \mid\cdot\mid_{p, \infty})$ are of the form $B_{r_i}(a) \times B_{r_j}(b)$, where $B_{r_i}(a)$ and $B_{r_j}(b)$ are balls in $(\mathbb{Z}_p, \mid\cdot\mid_p)$. 
\end{proposition}

\section{$\rho$-orderings}
\noindent Some of Keith's definitions and results about extending p-orderings to compact ultrametric spaces. 

\begin{definition}
	(\textbf{$\rho$-ordering})(KJ, definition 1) Let $(M, \rho)$ be an ultrametric space and $S \subseteq M$ a compact subset. A $\rho$-ordering of $S$ is a sequence $\{a_i\}_{i=0}^\infty \subseteq S$ such that $a_0$ is arbitrary and for each $n > 0$, $a_n$ maximizes $\prod_{i=0}^{n-1} \rho(s,a_i)$ over $s \in S$.  
\end{definition}

\begin{example}
	Subsets of $(\mathbb{Z}_p, \mid \cdot \mid_p)$
	\begin{itemize}
		\item  Consider $\mathbb{Z}$ as a subset of the compact metric space $(\mathbb{Z}_p, \mid \cdot \mid_p)$ for any prime p. Let $\rho_p$ be the induced metric, $\rho_p(z_1, z_2) = \mid z_1 - z_2 \mid_p = p^{-\nu_p(z_1-z_2)}$. The natural ordering on the integers is a $\rho_p$-ordering of $\mathbb{Z}$ for any p (Bhargava).
		
		\item 	Consider $p^h\mathbb{Z}$ as a subset of $(\mathbb{Z}_p, \mid \cdot \mid_p)$ for any prime p and $h \in \mathbb{N}$. Let $\{a_i\}_{i=0}^{\infty}$ be any $\rho_p$-ordering of $\mathbb{Z}$, for example $\{i\}_{i=0}^\infty$. Then for $n>0$, $\{p^ha_i\}_{i=0}^\infty$, maximizes $\prod_{i=0}^{n-1} \rho_p(s,p^ha_i)$ over $s \in p^h\mathbb{Z}$	since $\prod_{i=0}^{n-1} \rho_p(s,p^ha_i) = \prod_{i=0}^{n-1} \rho_p(p^hz,p^ha_i) = \prod_{i=0}^{n-1} p^{-h-\nu_p(z-a_i)}$ for some $z \in \mathbb{Z}, s= p^hz$. Then $\{p^ha_i\}_{i=0}^\infty$ is a $\rho_p$-ordering of $p^h\mathbb{Z}$. For example, $\{ip^h\}_{i=0}^\infty$ is a $\rho_p$-ordering of $p^h\mathbb{Z}$.
	\end{itemize}
\end{example}

\begin{example}
Subsets of $(\mathbb{Z}_p \times \mathbb{Z}_p, \mid \cdot \mid_{p, \infty})$
	\begin{itemize}
		\item 	Consider $\mathbb{Z} \times \mathbb{Z}$ as a subset of $(\mathbb{Z}_p \times \mathbb{Z}_p, \mid \cdot \mid_{p,\infty})$. Recall that \[\rho_{p,\infty}(u,v) = \rho_{p,\infty}((x_1,y_1),(x_2,y_2))= \max(\mid x_1 - x_2 \mid_p, \mid y_1 - y_2 \mid_p)\] \[= \max(p^{-\nu_p(x_1-x_2)},p^{-\nu_p(y_1-y_2) })\] with $x_i, y_i \in \mathbb{Z}$.  Let $\{a_i\}_{i=0}^{\infty}$ be a $\rho_p$-ordering for $\mathbb{Z}$. Then for $n>0$,  $(a_n, a_n)$ will maximize \[\prod_{i=0}^{n-1} \rho_{p,\infty} ((z_1, z_2), (a_i, a_i)) = \prod_{i=0}^{n-1} \max(p^{-\nu_p(z_1 - a_i)},p^{-\nu_p(z_2 - a_i)})\] since $\{a_i\}_{i=0}^{\infty}$ was chosen to be a $\rho_p$-ordering for $\mathbb{Z}$. Then $\{(a_i, a_i)\}_{i=0}^{\infty}$ will be a $\rho_{p, \infty}$-ordering of $\mathbb{Z} \times \mathbb{Z}$	
		
		
		\item 	Consider $p^h\mathbb{Z} \times p^k\mathbb{Z}$ as a subset of $(\mathbb{Z}_p \times \mathbb{Z}_p, \mid \cdot \mid_{p,\infty})$ and assume $k > h$. Then for $(u,v) \in p^h\mathbb{Z} \times p^k\mathbb{Z}, (u,v) = ((p^hx_1,p^ky_1), (p^hx_2, p^ky_2))$, with $x_i,y_i \in \mathbb{Z}$, and \[\rho_{p,\infty}(u,v) =  \max(p^{-\nu_p(p^h(x_1-x_2))},p^{-\nu_p(p^k(y_1-y_2)) })\]\[ = \max(p^{-h-\nu_p(x_1-y_1)}, p^{-k-\nu_p(x_2-y_2)})\]   Let $\{a_i\}_{i=0}^{\infty}$ be a $\rho_p$-ordering for $\mathbb{Z}$. Then for $n>0$,  $(p^ha_n, p^ka_n)$ will maximize $\prod_{i=0}^{n-1} \rho_{p,\infty} ((s_1, s_2), (p^ha_i, p^ka_i))$ over $s=(s_1,s_2) \in p^h\mathbb{Z} \times p^k\mathbb{Z}$ since \[\prod_{i=0}^{n-1} \rho_{p,\infty} ((s_1, s_2), (p^ha_i, p^ka_i)) = \prod_{i=0}^{n-1} \max(p^{-\nu_p(s_1 - p^ha_i)},p^{-\nu_p(s_2 - p^ka_i)}) \] \[ =\prod_{i=0}^{n-1} \max(p^{-\nu_p(p^h(z_1 - a_i))}, p^{-\nu_p(p^k(z_2 - a_i))})\]\[=\prod_{i=0}^{n-1} \max(p^{-h-\nu_p(z_1 - a_i)}, p^{-k-\nu_p(z_2 - a_i)})\] and $\{a_i\}_{i=0}^{\infty}$ was chosen to be a $\rho_p$-ordering for $\mathbb{Z}$. Then $\{(p^ha_i, p^ka_i)\}_{i=0}^{\infty}$ will be a $\rho_{p, \infty}$-ordering of ${p^h\mathbb{Z} \times p^k\mathbb{Z}}.$	
		
	\end{itemize}
\end{example}


\begin{definition}
	(\textbf{$\rho$-sequence})(KJ, definition 1)  Let $(M, \rho)$ be an ultrametric space and $S \subseteq M$ a compact subset. The $\rho$-sequence of $S$ is the sequence $\{1, \{\prod_{i=0}^{n-1} \rho(a_n,a_i)\}_{n=1}^{\infty}\}	$, where the $a_i$ are the elements of any $\rho$-ordering of $S$.
\end{definition}

\begin{example}
	With the notation of Example 1.1, the $\rho_p$-sequence of $\mathbb{Z}$ for any prime p is $\{1, \{\prod_{i=0}^{n-1} \rho(n,i)\}_{n=1}^{\infty}\}	= \{1, \{\prod_{i=0}^{n-1} p^{-\nu_p(n-i)}\}_{n=1}^{\infty}\}= \{1, \{p^{-(\nu_p(n) + \nu_p(n-1) +...+ \nu_p(1))}\}_{n=1}^{\infty}\}= \{1, \{p^{-\nu_p(n!)}\}_{n=1}^{\infty}\}$.
\end{example}	

\begin{example}
	With the notation of Example 1.2, the $\rho_{p, \infty}$-sequence of $\mathbb{Z} \times \mathbb{Z}$ for any prime p is ${}$
\end{example}	


\begin{proposition}
	(KJ, lemma 2) For $S$, a compact subset of a ultrametric space $(M, \rho)$, the $\rho$-sequence of $S$ is well-defined.
\end{proposition}

\begin{definition}
	(\textbf{valuative capacity})(KJ, definition 8) Let $S$ be a compact subset of an ultrametric space $(M, \rho)$ and let $\{\gamma(n)\}_{n=0}^{\infty}$ be the $\rho$-sequence of $S$. The valuative capacity of $S$ is $$\omega(S) = \lim_{n\to\infty} \gamma(n)^{1/n}$$
\end{definition}

\begin{example}
	With the notation of Example 1.1, $\omega(\mathbb{Z}) =\lim_{n\to\infty} \gamma(n)^{1/n} = \lim_{n\to\infty} [p^{-\nu_p(n!)}]^{1/n} = \lim_{n\to\infty} p^{-\frac{\nu_p(n!)}{n}} = p^{\frac{-1}{p-1}} = p^{1-p}.$ 
\end{example}

\begin{example}
	product space
\end{example}


\noindent We see through the examples of this section that at least for p-adic integers, the $\rho$-ordering, and as a direct consequence, the $\rho$-sequence and valuative capacity of a set, are well-behaved with respect group operations and taking products. In the next section, we formalize this observation and show that it holds for any compact ultrametric space.