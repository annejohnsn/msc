\section{Computation of the capacity of some sets\\ (F\&P, section 5)}
\subsection*{Setup}
Let $A=\{0,1,..,d-1\}$ be a finite alphabet and $A^{\mathbb{N}}$ be the collection of infinite sequenes with values in $A$. Note $A^{\mathbb{N}}$ is a Cantor set, so it is perfect, nowhere dense, and compact.  \\

\begin{framed}All of these are properties shared by limit sets \end{framed}

A basis for the topology is given by the cylinder set: take countably many copies of $\{0,1,...,d-1\}$ where each copy has the discrete topology. Let $p \geq d$ be a prime number and let $\phi$ be the canonical embedding of $A^\mathbb{N}$ into $\mathbb{Z}_p$ via the following continuous map: \[ \phi: A^{\mathbb{N}} \rightarrow \mathbb{Z}_p \text{ by } (x_n)_{n\geq0} \mapsto \sum_{k=0}^\infty x_kp^k\]

\begin{lemma*} (F\&P Lemma 5.1)\\ Let $w_1,w_2,\ldots,w_s$ be $s\geq 2$ words with the same length $l$ such that all the first letters ar distinct. Let $X \subset A^{\mathbb{N}}$ be the set of sequences such that any factor is a factor of a concatenation of the words $w_1,w_2,\ldots,w_s$. Then the set $E := \phi(X) \subset \mathbb{Z}_p$ satisfies: \[E=\cup_{i=1}^s x_i +p^l E \text{,   with } x_i=\phi(w_i0^\infty)\]
It is a regular compact set and its valuative capacity is \[L_p(E) = \frac{l}{s-1}\] Notice that this provides examples of sets with empty interiors but with positive capacities.
\end{lemma*}
\newpage
\begin{example}
$w_1=0, w_2=2,  A=\{0,1,2\}, p=d=3$\\
Then $\{x_n\}_{n\geq0} \in X$ if each term in  $\{x_n\}_{n\geq0}$ is either $0$ or $2$, so that\\  $E = \phi(X) \subset \mathbb{Z}_p$ is exactly the Cantor set. We have \[E=0 + 3E \cup 2 + 3E \text{ and }\]  \[L_p(E) = \frac{1}{2-1} =1\] 
(so we see that the Cantor set is self-similar and has capacity 1)
\end{example}

\begin{framed}
Note that:\newline $0 + 3E = \{3z \in \mathbb{Z}_3; z = \sum_{i=0}^\infty a_ip^i, a_i \in {0,2}\} =\\
\{z \in \mathbb{Z}_3; z = \sum_{i=1}^\infty a_ip^i, a_i \in {0,2}\} =\\
\{z \in \mathbb{Z}_3; ord(z) \geq 1 \} \cap E=\\
\{z \in \mathbb{Z}_3; \mid z \mid_3 \leq \frac{1}{3}\} \cap E = \\
\{z \in \mathbb{Z}_3; \mid 0- z \mid_3 \leq \frac{1}{3}\} \cap E = B_{\frac{1}{3}}(0) \cap E$.
\\\\and:\\ 
$2 + 3E = \{2 + 3z \in \mathbb{Z}_3; z = \sum_{i=0}^\infty a_ip^i, a_i \in {0,2}\} =\\
\{2 + z \in \mathbb{Z}_3; z = \sum_{i=1}^\infty a_ip^i, a_i \in {0,2}\} =\\
\{2 + z \in \mathbb{Z}_3; ord(z) \geq 1 \} \cap E=\\
\{z \in \mathbb{Z}_3; \mid 2 - z \mid_3 \leq \frac{1}{3}\} \cap E = \\
B_{\frac{1}{3}}(2) \cap E$
\\ and so $E = B_{\frac{1}{3}}(0)\cap E \cup B_{\frac{1}{3}}(2) \cap E$ 

\end{framed}