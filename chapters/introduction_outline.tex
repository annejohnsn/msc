
In the course of developing a generalized factorial function, Bhargava introduced the notion of $p-$orderings of a Dedekind domain \cite{mb1, mb2}, a combinatorial concept which, along with his generalized factorial, provided deep and perhaps unexpected results in number theory. The concepts laid down in these papers have enriched the theory of integer-valued polynomials \cite{mb3} (also KJ) and have also provided a natural framework to extend many classical results in analysis to a $p-$adic setting, such as polynomial approximation and mapping theorems \cite{mb1, mb2,mb3}.\\

In this thesis, we examine how a tool based on $p-$orderings can extend another concept from classical analysis, namely the \textit{valuative capacity} of a set, to non-archimedean settings.