As a first point of departure, a natural space to consider is the product space of ultrametric spaces, for example $\mathbb{Z}^n$ (or $\mathbb{Z}_p^n$ or $\mathbb{Q}_p^n$), for some $n >1$. If we restrict our attention to bounded subsets, then a natural candidate for an ultrametric on the product space is the $L_\infty$ metric, given by
\[ \rho_\infty(x,y) = \rho_\infty((x_1,x_2,\ldots),(y_1,y_2, \ldots)) = sup_{i} \{\rho(x_i, y_i)\}\] where $\rho$ is the metric from the base space. In fact, since we have only defined valuative capacity for compact subsets of an ultrametric spaces, there is no loss of generality by restricting our metric to bounded spaces.  We also see that no problems arise in letting both $M$ and $\rho$ vary between components of the space, as long as each $M_i$ remains bounded and each $\rho_i$ is an ultrametric. 

\begin{proposition*}
Let $(M_i, \rho_i)$ for $i$ in some finite or countably infinite index set $I$ be a collection of metric spaces and suppose $\rho_i$ is a bounded ultrametric for all $i$. Then $(M,\rho_\infty)$ is an ultrametric space, where $M=M_1 \times M_2 \times M_3 \times \ldots$ and $\rho_\infty$ is the  $L_\infty$ metric described above.
\end{proposition*}

\begin{proof}
Let $(M, \rho_\infty)$ be the product of ultrametric spaces as above and let $x$ and $y$ be two points in the space. Clearly, $\rho_\infty(x,y) \geq 0$ since each $\rho_i(x_i,y_i) \geq 0$, and $\rho_\infty(x,y) = 0 \iff \rho_i(x_i,y_i) =0,\forall i \iff x_i=y_i, \forall i \iff x=y$. The fact that $\rho_\infty$ is symmetric is also an easy consequence of the fact that each $\rho_i$ is symmetric since  $\rho_i(x_i, y_i) = \rho_i(y_i, x_i)$ implies $sup_{i}\{\rho_i(x_i, y_i)\} = sup_{i}\{\rho_i(y_i, x_i)\}$. To see that $\rho_\infty$ is an ultrametric, note that if $z=z_i$ is any other point of $M$, then
\begin{align*}
    \rho_\infty(x, y) = sup_i\{\rho_i(x_i,y_i)\} && \\
    \leq  sup_i\{max(\rho_i(x_i,z_i),\rho_i(y_i,z_i))\} && \text{ since each $\rho_i$ is an ultrametric } \\
    \leq max(\rho_i(x_i,z_i)\rho_i(y_i,z_i)), \forall i && \\
    \leq max(sup_i\{\rho_i(x_i,z_i)\}, sup_i\{\rho_i(y_i,z_i)\} && \\
    = max(\rho_\infty(x,z),\rho_\infty(y,z))  
\end{align*}

\end{proof}

We show a few quick results ultrametric spaces formed as product spaces, which allows us to quickly calculate the valuative capacity of a few subsets. 

\begin{proposition*}
Suppose $(M,\rho_\infty)$ is the product of ultrametric spaces $(M_i, \rho_i)$ and each $M_i$ is a topological group with operation $+$. Then $\rho_\infty$ is (left) translation invariant if each $\rho_i$ is, in which case valuative capacity is also (left) translation invariant.
\end{proposition*}

\begin{proof}
Let $(M,\rho_\infty)$ be as above. Suppose also that \[\rho(x_i,y_i) = \rho(s_i + x_i, s_i +y_i), \forall s_i, x_i, y_i \in M_i, \forall i.\] that is, suppose each $\rho_i$ is (left) translation invariant. Then,  
\[
\rho_\infty(s +x, s+y)  
= sup_i\{\rho(s_i +x_i, s_i + y_i)\} 
= sup_i\{\rho(x_i, y_i)\}
= \rho_\infty(x,y).
\] so that $\rho_\infty$ is translation invariant.  Proposition $xyz$ implies valuative capacity is as well. 
\end{proof}


\begin{proposition*}
Suppose $(M,\rho_\infty)$ is the product of ultrametric spaces $(M_i, \rho_i)$ and each $M_i$ is in fact a normed vector space (with each $\rho_i$ being the metric derived from the norm on $M_i$). Then $\rho_\infty$ is multiplicative if the norm producing each $\rho_i$ is multiplicative, in which valuative capacity is as well.
\end{proposition*}

\begin{proof}
Similar.
\end{proof}

\begin{example}
	Let $(\mathbb{Z}_p \times \mathbb{Z}_p, \rho_{p,\infty})$ be the metric space with elements $\{(x,y)\mid x,y \in \mathbb{Z}_p\}$ and metric $\rho_{p,\infty}((x_1,x_2), (y_1,y_2)) = \max(\rho_p(x_1, y_1)), \rho_p(x_2, y_2))$, where $\rho_p$ is the p-adic metric for some fixed prime $p$. Since $\rho_p$ is translation invariant and multiplicative in $\mathbb{Z}_p$, valuative capacity is also translation invariant and multiplicative in  $(\mathbb{Z}_p \times \mathbb{Z}_p, \rho_{p,\infty})$.
\end{example}

\begin{example}
	Let $(\mathbb{Z}_{p_1} \times \mathbb{Z}_{p_2}, \rho_{p,\infty})$ be the metric space with elements $\{(x,y)\mid x \in \mathbb{Z}_{p_1}, y \in \mathbb{Z}_{p_2}\}$ for two distinct primes, $p_1 \neq p_2$, and metric $\rho_{p,\infty}((x_1,x_2), (y_1,y_2)) = \max(\rho_{p_1}(x_1, y_1)), \rho_{p_2}(x_2, y_2))$, where $\rho_{p_i}$ is the p-adic metric. Since each $\rho_{p_i}$ is translation invariant and multiplicative in $\mathbb{Z}_{p_i}$, valuative capacity is also translation invariant and multiplicative in  $(\mathbb{Z}_{p_1} \times \mathbb{Z}_{p_2}, \rho_{p,\infty})$.
\end{example}


\section*{ \textbf{$n$-fold products}}
What is the valuative capacity of  $(\mathbb{Z}_p \times \mathbb{Z}_p, \rho_{p,\infty})$  from the example above? Suppose $p=2$.  Using translation invariance, scaling and subaddivity, we can compute the result by first noting that we can write $\mathbb{Z}_2 \times \mathbb{Z}_2$ as a union, as below,
\[
\mathbb{Z}_2 \times \mathbb{Z}_2 = (2\mathbb{Z}_2 \times 2\mathbb{Z}_2) \cup (2\mathbb{Z}_2 \times 2\mathbb{Z}_2 +1) \cup (2\mathbb{Z}_2+1 \times 2\mathbb{Z}_2) \cup (2\mathbb{Z}_2+1, 2\mathbb{Z}_2+1).
\]

Since the pairwise distances on the right-hand side are always $1 = diam(\mathbb{Z}_2 \times \mathbb{Z}_2)$, subadditivity implies that 

\[
\frac{1}{log(\omega(\mathbb{Z}_2 \times \mathbb{Z}_2))} \]
\[ = \frac{1}{log(\omega(2\mathbb{Z}_2 \times 2\mathbb{Z}_2))} + \frac{1}{log(\omega(2\mathbb{Z}_2 \times 2\mathbb{Z}_2 +1))} + \frac{1}{log(\omega(2\mathbb{Z}_2+1 \times 2\mathbb{Z}_2))} + \frac{1}{log(\omega(2\mathbb{Z}_2+1 \times 2\mathbb{Z}_2+1))}\]
\[ = \frac{4}{log(\|2\|_2 * \omega(\mathbb{Z}_2 \times \mathbb{Z}_2))} = \frac{4}{log(\frac{1}{2} * \omega(\mathbb{Z}_2 \times \mathbb{Z}_2))} =  \frac{4}{log(\frac{1}{2}) + log(\omega(\mathbb{Z}_2 \times \mathbb{Z}_2))}\]

 Taking logs base $2$, we have that 
\[\omega(\mathbb{Z}_2 \times \mathbb{Z}_2) = 2^{\frac{-1 + log_2(\omega(\mathbb{Z}_2 \times \mathbb{Z}_2))}{4}} =  2^{\frac{-1}{4}} 2^ {\frac{log_2(\omega(\mathbb{Z}_2 \times \mathbb{Z}_2))}{4}}
= 2^{\frac{-1}{4}}{(2^ {log_2(\omega(\mathbb{Z}_2 \times \mathbb{Z}_2))})}^{\frac{1}{4}} = 2^{\frac{-1}{4}}{\omega(\mathbb{Z}_2 \times \mathbb{Z}_2)}^{\frac{1}{4}} \]

so that ${\omega(\mathbb{Z}_2 \times \mathbb{Z}_2)}$ is a solution of the equation $x^4 - \frac{x}{2}$, for which there is a single real positive root, given by $2^{-1/3}$.\\


To compute the valuative capacity for a $2$-fold product for an arbitary prime $p$, note that we can always decompose $\mathbb{Z}_p \times \mathbb{Z}_p$ into a union of $p^2$ sets each of the form $\{p\mathbb{Z}_p+s \times p\mathbb{Z}_p +t\}$ for $s,t \in (0,\ldots, p-1)$, and the pairwise distance between these sets will always be $1 = diam(\mathbb{Z}_p \times \mathbb{Z}_p)$ (to see this, either note that we can always find co-prime elements, or note that each set is an closed ball of radius $1/p$ centred at (s,t) and so the distance between them must be greater than $1/p$, and $1$ is the only possible distance greater than $1/p$ in $\mathbb{Z}_p \times \mathbb{Z}_p$).  Then, we combine our tools as before to obtain the equation,

\[\frac{1}{log(\omega(\mathbb{Z}_p \times \mathbb{Z}_p))} = \frac{p^2}{log(\|p\|_p * \omega(\mathbb{Z}_p \times \mathbb{Z}_p))} =  \frac{p^2}{log(1/p * \omega(\mathbb{Z}_p \times \mathbb{Z}_p))}    \]

In turn, taking logs base p, we have 


\[ \omega(\mathbb{Z}_p \times \mathbb{Z}_p) = p^{\frac{-1}{p^2}} \omega(\mathbb{Z}_p \times \mathbb{Z}_p)^{\frac{1}{p^2}}  \]

So that $\omega(\mathbb{Z}_p \times \mathbb{Z}_p)$ is a solution of the equation $x^{p^2} - \frac{x}{p} = x(x^{p^2-1} - \frac{1}{p})$ over $\mathbb{R}$ and since $\mathbb{R}$ is a division ring, this means the positive solutions are given by solving $x^{p^2-1}-\frac{1}{p}$. Solutions of this equation are of the form $p^{\frac{-1}{p^2-1}}$ times a $p^2-1$ root of unity, and so there is exactly one positive, real solution, namely $p^{\frac{-1}{p^2-1}}$ itself. Then the valulative capacity of the entire product space $\mathbb{Z}_p \times \mathbb{Z}_p$ is $p^{\frac{-1}{p^2-1}}$. In fact, from here it is not hard to see that by taking the $n$-fold product, we would end up with the same equation except that the exponent of $p$ would become $n$ rather than $2$. We arrive at the following result:

\begin{proposition*}
Let $M=(\mathbb{Z}_p^n, \rho_{p, \infty})$ be the ultrametric space with points equal to the $n$-fold product of $\mathbb{Z}_p$ (for $n < \infty$) for some fixed prime $p$. The valuative capacity of $M$ is  $(\frac{1}{p})^{\frac{1}{p^n-1}}$.
\end{proposition*}

\begin{proof}
Above.
\end{proof}

Taking $n=1$, we see that this agrees with the valuative capacity of $\mathbb{Z}_p$ computed in the last chapter. \\

We end this section with two observations on the results above.  First, recall that in computing the valuative capacity of these spaces, we were ultimately reduced to finding solutions to polynomials of the form $x^{p^n} - \frac{x}{p}$ for some $n$ and for some $p$. The first observation is that these polynomials are $\mathbb{Z}$-valued on $p\mathbb{Z}$, that is, they are elements of $Int(p\mathbb{Z},\mathbb{Z})$. We might ask then, what sort of polynomials would arise in finding the valuative capacity of spaces such as $(\mathbb{Z}_2 \times \mathbb{Z}_3, \rho_\infty)$ or in computing the valuative capacity of infinite product spaces, such as $\mathbb{Z}_p \times \mathbb{Z}_p \times \mathbb{Z}_p \times \ldots$ for either some fixed prime $p$ or over each prime. \\

Secondly, we observe the asymptotic behavior of capacity in these spaces. For a fixed prime $p$,  $(\frac{1}{p})^{\frac{1}{p^n-1}}$ is an monotone, increasing sequence in $n$ with $ lim_{n\to\infty} (\frac{1}{p})^{\frac{1}{p^n-1}} =  1$. For fixed $n$, the sequence in $p$ is also montone, increasing, again with  $ lim_{p\to\infty} (\frac{1}{p})^{\frac{1}{p^n-1}}=1$. In both cases, the limiting value is equal to the diameter of space. Indeed, we can observe that the sequence $\{(0,0,\ldots), (1,0,\ldots), (0,1,\ldots), \ldots\}$, in which the first element has only zeros and the $n$-th element has a single $1$ in the $(n-1)$-th component, is a $\rho-$ordering for both  $(\mathbb{Z}_p \times \mathbb{Z}_p \times \ldots, \rho_{p,\infty})$  and  $(\mathbb{Z}_2 \times \mathbb{Z}_3 \times \ldots, \rho_{P,\infty})$, since the distance between elements in this sequence (in either metric space) is always $1$. If we could show that these spaces are compact, this would gives a valuative capacity of $lim_{n\to\infty} (1^n)^{(1/n)} =1$ for both spaces. We explore this more in the following section.\\

\section*{Product topology}
In considering the product space of ultrametric spaces, we may wonder whether the chosen metric also gives back the product topology on the space. For products formed by taking some finite number of copies, the answer is positive. We give the necessary background and show this fact, adapting the proof in Munkres (20.3) to the case of ultrametric spaces. 

\begin{definition-proposition*} (Munkres)
Suppose $X_i$, for $i$ in some index set $I$, is a family of topological spaces.  Let $\pi_j: \prod_{i \in I} X_i \rightarrow X_j$ be the map given by projection onto the $j$-th component, that is $\pi_j (x) = \pi_j ((x_i)_{i \in I}) = x_j$. For each $j \in I$, let $\mathcal{S}_j$ be the collection \[\mathcal{S}_j = \{\pi^{-1}_j (U_j)\mid U_j \text{ open in } X_j\}\] Let $\mathcal{S}$ be the union of the $\mathcal{S}_j$ over $j \in I$, $\mathcal{S}= \cup_{j \in I} \mathcal{S}_j$. Then $\mathcal{S}$ is a subbasis that generates a topology on  $\prod_{i \in I} X_i$ called the \textbf{product topology}.\\

\end{definition-proposition*}


 The basis, $\mathcal{B}$, generated by $\mathcal{S}$ in the definition above is the set of all finite intersections of elements in $\mathcal{S}$. That is $B \in \mathcal{B}$ if there exists $S_1, S_2, \ldots, S_n$ in $\mathcal{S}$ such that $B = S_1 \cap S_2 \cap \ldots S_n$.  A useful description of the basis for the product topology also appears in Munkres, as below:

\begin{proposition*} (Munkres 19.2)
Suppose $X_i$, for $i$ in some index set $I$, is a family of topological spaces and denote by $\mathcal{B}_i$ the basis for the topology on $X_i$. Let 

\begin{align*}
\mathcal{B}_P = \prod_{i \in I} B_i, & \text{ for }  B_i \in \mathcal{B}_i \text { and } B_i = X_i \text{ for all but finitely-many } i \in I. 
\end{align*}
then $\mathcal{B}_P$ is a basis for the product topology on $\prod_{i \in I} X_i$.\\

\end{proposition*}

We can now show that the topology induced by the $L_\infty$ metric described above agrees with the product topology for finite products.

\begin{proposition*}
Let $M=(M_{1} \times M_{2} \times \ldots \times M_{n},\rho_\infty)$ be a finite product of bounded, ultrametric spaces and let $\rho_\infty$ be the metric described above.  Then the topology induced by $\rho_\infty$ coincides with the product topology on $M_{1} \times M_{2} \times \ldots \times M_{n}$.
\end{proposition*}

\begin{proof}
Let $\mathcal{T}_{\rho_\infty}$ be the topology on $M_{1} \times M_{2} \times \ldots \times M_{n}$ induced by $\rho_\infty$ and let $\mathcal{B}_{\rho_\infty}$ be the basis for this topology. Let $\mathcal{T}_{P}$ be the product topology with basis $\mathcal{B}_P$. We show $\mathcal{T}_{P} \subset \mathcal{T}_{\rho_\infty}$ and vice versa. For this, it is equivalent (Munkres 13.3) to show that for $z \in  M_{1} \times M_{2} \times \ldots \times M_{n}$ and $B \in \mathcal{B}_{P}$ containing $z$, there is a basis element $B' \in \mathcal{B}_{\rho_\infty}$ such that $z \in B' \subset B$, and vice versa.\\

So let  $z \in M_{1} \times M_{2} \times \ldots \times M_{n}$ and suppose $B \in \mathcal{B}_{P}$ contains $z$. Since $B$ is in $\mathcal{B}_{P}$, $B$ is of the form $B_{r_1}(z_1) \times B_{r_2}(z_2) \times \ldots \times B_{r_n}(z_n)$ (since the choice of centres is arbitrary in an ultrametric spaces, we may choose the components of $z$ as the centres without loss of generality). Let $r=\min \{r_i\}$ for $i \in 1,\ldots, n$. Then let $B'$ be the ball $B_r(z)$ in $ \mathcal{B}_{\rho_\infty}$. Clearly, $z \in B_r(z)$ and since $r \leq r_i$, $\forall i$, $B_r(z) = B_r(z_1) \times B_r(z_2) \times \ldots \times B_r(z_n) \subset  B_{r_1}(z_1) \times B_{r_2}(z_2) \times \ldots \times B_{r_n}(z_n) =B$.\\

Conversely, suppose $A \in \mathcal{B}_{\rho_\infty}$ and let $y \in A$. To find $A' \in \mathcal{B}_{P}$ such that $y \in A'$ and $A' \subset A$, simply note that $A$ itself is in $\mathcal{B}_{P}$.

\end{proof}

We are now naturally left to ask whether the product topology on \textit{infinite} products of ultrametric spaces coincides with the $L_\infty$ metric. In this case, as in the analogous case of infinite copies of $\mathbb{R}$ and a uniform metric, the answer is negative (at least in general). Forunately, the metric that realizes the product topology on infinite copies of $\mathbb{R}$ can be adapted to the case of ultrametric spaces. We adapt to the proof of Munkres (20.5) to the case of infinite products of ultrametric spaces.

\begin{proposition*}
Suppose $\textbf{M} = M_{1} \times M_{2} \times \ldots$ is an infinite collection of metric spaces, each with an ultrametric $\rho_i$ which is bounded by 1, that is suppose $\rho_i(x_i,y_i) \leq 1$, for all $x_i, y_i \in M_i$ and for all $i$. Define a metric $d$ on $\textbf{M}$ as follows: \[d(\textbf{x},\textbf{y}) = sup\{\frac{\rho_i(x_i,y_i)}{i}\}\]

Then $d$ is an ultrametric and induces the product topology on $\textbf{M}$.
\end{proposition*}

\begin{proof}
We see that $d$ inherits symmetry, injectivity and non-negativity from the requirement that each $\rho_i$ is a metric, just as $\rho_\infty$ did.  To see that $d$  satisfies the strong triangle inequality, define a new metric $\rho_i'$ by $\rho_i'(x,y)=\frac{\rho_i(x_i,y_i)}{i}, \forall i$. Then  $\rho_i'$ is an ultrametric, since $\rho_i(x,y) \leq \max(\rho_i(x,z), \rho_i(y,z))$ implies  $\frac{\rho_i(x,y)}{n} \leq \max(\frac{\rho_i(x,z)}{n}, \frac{\rho_i(y,z)}{n})$ for any $n \in \mathbb{N}$. Then we can view $d$ as the $L_\infty$ metric on the spaces $(M_i, \rho_i')$, and so $d$ will be an ultrametric as shown in the first proposition of this section.\\

Now we show $d$ induces the product topology. We first show that metric topology induced by $d$ is finer than the product topology. Let \[B=B_r^{\textbf{M}}(\textbf{z}) =  B_r^{M_1}(z_1) \times B_{2r}^{M_2}(z_2)  \times B_{3r}^{M_3}(z_3) \times \ldots  \] be a basis open in the metric topology. We must find a basis open $B' \ni z$ in the product topology such that $B' \subseteq B$. Let $N \in \mathbb{N}$ be such that $\frac{1}{N} < r$. Then let $B'$ be the basis open element \[B' = B_r^{M_1}(z_1) \times B_r^{M_2}(z_2) \times \ldots \times B_r^{M_N}(z_N) \times M_{N+1} \times M_{N+2} \times \ldots \] in the product topology. Suppose $\textbf{y} \in B'$. We must show $\textbf{y} \in B$, i.e., $d(\textbf{z}, \textbf{y}) < r$. Note that for all $i \geq N$, 
\[\frac{\rho_i(z_i, y_i)}{i} \leq \frac{1}{N}\]

which means 

 \[d(\textbf{z},\textbf{y}) = \sup\{\frac{\rho_i(z_i,y_i)}{i}\} \leq \max\{\frac{\rho_1(z_1,y_1)}{1}, \frac{\rho_2(z_2,y_2)}{2}, \ldots, \frac{\rho_N(z_N,y_N)}{N}, \frac{1}{N}\}\]

and since $N$ was chosen so that $\frac{1}{N} < r$ and $B'$ was chosen to have balls of radius $r$ in the first $N$ components, we must have $d(\textbf{z}, \textbf{y}) < r$.

Conversely, $\ldots$
\end{proof}

From now on, we refer to the metric $d$ above as the \textbf{product metric}. An important consequence of the fact that $d$ achieves the product topology is that Tychnoff's theorem then guarantees that product spaces formed with this metric will be compact, infinite or otherwise. As a result, we can now ask directly about the valuative capacity of some infinite product spaces. We consider two examples. 

\begin{example}
	Let $(\mathbb{Z}_p \times \mathbb{Z}_p \times \mathbb{Z}_p \times \ldots, d)$ be the metric space formed by taking the product of $(\mathbb{Z}_p, \rho_p)$ for some fixed prime $p$ and let $d$ be the product metric.
\end{example}

\begin{example}
	Let $(\mathbb{Z}_2 \times \mathbb{Z}_3 \times \mathbb{Z}_5 \times \ldots, \rho_{P,\infty})$ be the metric space formed by taking the product of $(\mathbb{Z}_p, \rho_p)$ for every prime $p$ and let $d$ be the product metric.
\end{example}

So far we have two methods for computing valuative capacity. Either we can find a useful decomposition that allows us to apply the subadditivity formula, or we can find a $\rho-$ordering and then take the limit of its corresponding $\rho-$sequence.


\section*{Conclusion}
In this section, we considered the notion of valuative capacity in product spaces, that is, spaces formed by taking copies of ultrametric spaces. In the following sections, we consider  vaulative capacity in spaces formed by adding points, that is extension fields, or by both taking copies and adding (a distinguished) point, as in projective spaces. For these purposes, it will be more productive to start working over the field $\mathbb{Q}_p$, instead of $\mathbb{Z}_p$.




































