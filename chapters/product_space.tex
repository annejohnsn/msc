We consider now an application on the above. A natural space to consider is the product space of ultrametric spaces, for example $\mathbb{Z}^n$, for some $1 < n < \infty$. A natural candidate for an ultrametric on a finite product space is given by
\[ \rho_\infty(x,y) = \rho_\infty((x_1,x_2,\ldots,x_n),(y_1,y_2, \ldots, y_n)) = \max_{i} \{\rho(x_i, y_i)\}\] where $\rho$ is the metric from the base space.  We also see that no problems arise in letting both $M$ and $\rho$ vary between components of the space, as long as each $\rho_i$ is an ultrametric.\\

\begin{proposition}
Let $(M_i, \rho_i)$ for $i$ in some finite index set, $I$, be a collection of metric spaces and suppose $\rho_i$ is an ultrametric for all $i$. Then $(M,\rho_\infty)$ is an ultrametric space, where $M=M_1 \times M_2 \times M_3 \times \ldots \times M_n$ and $\rho_\infty$ is the metric described above.
\end{proposition}

\begin{proof}
Let $(M, \rho_\infty)$ be the product of ultrametric spaces as above and let $x$ and $y$ be two points in the space. Clearly, $\rho_\infty(x,y) \geq 0$ since each $\rho_i(x_i,y_i) \geq 0$, and $\rho_\infty(x,y) = 0 \iff \rho_i(x_i,y_i) =0,\forall i \iff x_i=y_i, \forall i \iff x=y$. The fact that $\rho_\infty$ is symmetric is also an easy consequence of the fact that each $\rho_i$ is symmetric since  $\rho_i(x_i, y_i) = \rho_i(y_i, x_i)$ implies $\max_{i}\{\rho_i(x_i, y_i)\} = \max_i\{\rho_i(y_i, x_i)\}$. To see that $\rho_\infty$ is an ultrametric, note that if $z=z_i$ is any other point of $M$, then
\begin{align*}
    \rho_\infty(x, y) = \max_i\{\rho_i(x_i,y_i)\} && \\
    \leq \max_i\{\max(\rho_i(x_i,z_i),\rho_i(y_i,z_i))\} && \text{ since each $\rho_i$ is an ultrametric } \\
    \leq \max(\max_i\{\rho_i(x_i,z_i)\}, \max_i\{\rho_i(y_i,z_i)\})  \\
    = \max(\rho_\infty(x,z),\rho_\infty(y,z))  
\end{align*}
%* Let $M= max(sup_i(\{a_i\}, sup_j(\{b_j\})))$, then $M \geq a_i, \forall i$ and $M \geq b_i, \forall i$, so $M \geq max(a_i,b_i), \forall i$, hence $M \geq sup_i(max(a_i,b_i)$.
\end{proof}

We first show that translation invariance carries over into product spaces under the expected conditions.\\ 

\begin{proposition}
Suppose $(M,\rho_\infty)$ is the product of ultrametric spaces $(M_i, \rho_i)$ and each $M_i$ is a topological group with operation $+_i$. Let $+$ denote the operation on $M$ given by $s+x = (s_1 + _1 x_1, s_2 +_2 x_2, \ldots, s_n +_nx_n)$ for $s = (s_1,\ldots,s_n)$and $x =(x_1,\ldots,x_n)$ in $(M, \rho_\infty)$. Then $\rho_\infty$ is (left) translation invariant under $+$ if each $\rho_i$ is (left) translation invariant under $+_i$, in which case valuative capacity is also (left) translation invariant.
\end{proposition}

\begin{proof}
Let $(M,\rho_\infty)$ be as above. Suppose also that \[\rho_i(x_i,y_i) = \rho_i(s_i +_i x_i, s_i +_iy_i), \forall s_i, x_i, y_i \in M_i, \forall i.\] that is, suppose each $\rho_i$ is (left) translation invariant. Then,  
\[
\rho_\infty(s +x, s+y)  
= \max_i\{\rho_i(s_i +_ix_i, s_i +_i y_i)\} 
= \max_i\{\rho_i(x_i, y_i)\}
= \rho_\infty(x,y).
\] so that $\rho_\infty$ is translation invariant.  Proposition \ref{translation invariance} implies valuative capacity is as well. 
\end{proof}

In the next proposition, we show that scaling carries over to product space as well, although the conditions are now more restrictive. In contrast to the proposition above, here  we cannot allow the spaces to vary between components.\\

\begin{proposition}
Let $(m, \rho_N)$ be an ultrametric space, where $\rho_N$ is the metric induced by some norm $N$. Let $(M, \rho_{\infty})$ be the ultrametric space formed by taking products of $m$, along with the $\rho_\infty$ metric defined above.  Then if $\rho_N$ is multiplicative on $m$, $\rho_{\infty}$ is multiplicative on $M$, in the sense that $\rho_{\infty}(cx,cy) = \lvert c\rvert_{\rho_N} \text{ } \rho_\infty (x,y)$, for $c=(c,c,c,\ldots), x,y \in M$.
\end{proposition}

\begin{proof}
Let $M, \rho,$ and $\rho_{\infty}$ be as above. Then, 
\begin{align*}
\rho_\infty(cx, cy) && \\
= \max_i\{\rho_N(c_i x_i, c_i y_i)\} && \\
= \max_i\{\vert c \rvert_{\rho_N} \text{ }  \rho_N(x_i, y_i)\} && \\
= \vert c \rvert_{\rho_N} \text{ } \max_i\{\rho_N(x_i,y_i)\} && \\
= \vert c \rvert_{\rho_N} \text{ } \rho_\infty(x_i,y_i)
\end{align*}
\end{proof}

\begin{corollary}
Let $S$ be a subset of $(M, \rho_\infty)$, where $M$ is the product of an ultrametric space $(m, \rho_N)$, which is itself a normed vector space with a multiplicative norm inducing $\rho_N$. If $c=(c,c,c,\ldots)$ is an element of $M$ with constant value on each component, then $\omega(cS)=\lvert c \rvert_{\rho_N} \text{ }\omega(S)$.
\end{corollary}

\begin{proof} 
The result follows by noting that if $\{a_j\}_{j=0}^\infty$ is a $\rho_\infty$ ordering of $S$, then $\{ca_j\}_{j=0}^\infty$ is a $\rho_\infty$ ordering of $cS$.
\end{proof}

We now introduce two examples, the details of which make up the rest of this chapter.\\ 

\begin{example}
	Let $(\mathbb{Z}_p \times \mathbb{Z}_p, \rho_{p,\infty})$ be the metric space with elements $\{(x,y)\mid x,y \in \mathbb{Z}\}$ and metric $\rho_{p,\infty}((x_1,x_2), (y_1,y_2)) = \max(\rho_p(x_1, y_1)), \rho_p(x_2, y_2))$, where $\rho_p$ is the p-adic metric for some fixed prime $p$. Since $\rho_p$ is translation invariant and multiplicative, valuative capacity is also translation invariant and multiplicative in  $(\mathbb{Z}_p \times \mathbb{Z}_p, \rho_{p,\infty})$.
\end{example}

\begin{example}
	Let $(\mathbb{Z}_{p_1} \times \mathbb{Z}_{p_2}, \rho_{P,\infty})$ be the metric space with elements $\{(x,y)\mid x,y \in \mathbb{Z}\}$ and metric $\rho_{P,\infty}((x_1,x_2), (y_1,y_2)) = \max(\rho_{p_1}(x_1, y_1)), \rho_{p_2}(x_2, y_2))$, for two distinct primes, $p_1 \neq p_2$, where both $\rho_{p_i}$ are p-adic metrics. Since each $\rho_{p_i}$ is translation invariant in $\mathbb{Z}_{p_i}$, valuative capacity will be translation invariant  in  $(\mathbb{Z}_{p_1} \times \mathbb{Z}_{p_2}, \rho_{P,\infty})$; however, unlike the case of $p_1=p_2$, this space does not have a multiplicative property that allows for scaling.
\end{example}

What is the valuative capacity of  $(\mathbb{Z}_p \times \mathbb{Z}_p, \rho_{p,\infty})$  from the example above? Suppose $p=2$.  Using translation invariance, scaling and subaddivity, we can compute the result by first noting that we can write $\mathbb{Z}_2 \times \mathbb{Z}_2$ as a union, as below,\\
\[
\mathbb{Z}_2 \times \mathbb{Z}_2 = (2\mathbb{Z}_2 \times 2\mathbb{Z}_2) \cup (2\mathbb{Z}_2 \times 2\mathbb{Z}_2 +1) \cup (2\mathbb{Z}_2+1 \times 2\mathbb{Z}_2) \cup (2\mathbb{Z}_2+1, 2\mathbb{Z}_2+1).
\]

Since the pairwise distances on the right-hand side are always $1 = diam(\mathbb{Z}_2 \times \mathbb{Z}_2)$, subadditivity implies that \\

\[
\frac{1}{log(\omega(\mathbb{Z}_2 \times \mathbb{Z}_2))} \]
\[ = \frac{1}{log(\omega(2\mathbb{Z}_2 \times 2\mathbb{Z}_2))} + \frac{1}{log(\omega(2\mathbb{Z}_2 \times 2\mathbb{Z}_2 +1))} + \frac{1}{log(\omega(2\mathbb{Z}_2+1 \times 2\mathbb{Z}_2))} + \frac{1}{log(\omega(2\mathbb{Z}_2+1 \times 2\mathbb{Z}_2+1))}\]
\[ = \frac{4}{log(\lvert 2 \rvert_2 \cdot \omega(\mathbb{Z}_2 \times \mathbb{Z}_2))} = \frac{4}{log(\frac{1}{2} \cdot \omega(\mathbb{Z}_2 \times \mathbb{Z}_2))}\]% =  \frac{4}{log(\frac{1}{2}) + log(\omega(\mathbb{Z}_2 \times \mathbb{Z}_2))}\]
Then,
\[4\cdot log(\omega(\mathbb{Z}_2 \times \mathbb{Z}_2)) =log(\frac{\omega(\mathbb{Z}_2 \times \mathbb{Z}_2)}{2}) \]


% Taking logs base $2$, we have that 
%\[\omega(\mathbb{Z}_2 \times \mathbb{Z}_2) = 2^{\frac{-1 + log_2(\omega(\mathbb{Z}_2 \times \mathbb{Z}_2))}{4}} =  2^{\frac{-1}{4}} 2^ {\frac{log_2(\omega(\mathbb{Z}_2 \times \mathbb{Z}_2))}{4}}
%= 2^{\frac{-1}{4}}{(2^ {log_2(\omega(\mathbb{Z}_2 \times \mathbb{Z}_2))})}^{\frac{1}{4}} = 2^{\frac{-1}{4}}{\omega(\mathbb{Z}_2 \times \mathbb{Z}_2)}^{\frac{1}{4}} \]

so that ${\omega(\mathbb{Z}_2 \times \mathbb{Z}_2)}$ is a solution of the equation $x^4 - \frac{x}{2}$, for which there is a single real positive root, given by $2^{-1/3}$.\\


To compute the valuative capacity for a $2$-fold product for an arbitary prime $p$, note that we can always decompose $\mathbb{Z}_p \times \mathbb{Z}_p$ into a union of $p^2$ sets each of the form $\{p\mathbb{Z}_p+s \times p\mathbb{Z}_p +t\}$ for $s,t \in (0,\ldots, p-1)$, and the pairwise distance between these sets will always be $1 = diam(\mathbb{Z}_p \times \mathbb{Z}_p)$ (to see this, either note that we can always find co-prime elements, or note that each set is an closed ball of radius $1/p$ centred at (s,t) and so the distance between them must be greater than $1/p$, and $1$ is the only possible distance greater than $1/p$ in $\mathbb{Z}_p \times \mathbb{Z}_p$).  Then, we combine our tools as before to obtain the equation,\\

\[\frac{1}{log(\omega(\mathbb{Z}_p \times \mathbb{Z}_p))} = \frac{p^2}{log(\lvert p \rvert_p \cdot \omega(\mathbb{Z}_p \times \mathbb{Z}_p))} =  \frac{p^2}{log(\frac{1}{p} \cdot \omega(\mathbb{Z}_p \times \mathbb{Z}_p))}    \]

In turn, we have 


\[ \omega(\mathbb{Z}_p \times \mathbb{Z}_p)^{p^2} =  \frac{\omega(\mathbb{Z}_p \times \mathbb{Z}_p)}{p}  \]

So that $\omega(\mathbb{Z}_p \times \mathbb{Z}_p)$ is a solution of the equation $x^{p^2} - \frac{x}{p} = x(x^{p^2-1} - \frac{1}{p})$ over $\mathbb{R}$. Since $\mathbb{R}$ is a division ring, this means the positive solutions are given by solving $x^{p^2-1}-\frac{1}{p}$. Solutions of this equation are of the form $p^{\frac{-1}{p^2-1}}$ times a $p^2-1$ root of unity, and so there is exactly one positive, real solution, namely $p^{\frac{-1}{p^2-1}}$ itself. Then the valulative capacity of the entire product space $\mathbb{Z}_p \times \mathbb{Z}_p$ is $p^{\frac{-1}{p^2-1}}$. In fact, from here it is not hard to see that by taking the $n$-fold product, we would end up with the same equation except that the exponent of $p$ would become $n$ rather than $2$. We arrive at the following result:\\

\begin{proposition}
Let $M=(\mathbb{Z}_p^n, \rho_{p, \infty})$ be the ultrametric space with points equal to the $n$-fold product of $(\mathbb{Z}, \rho_p)$ (for $n < \infty$) for some fixed prime $p$. The valuative capacity of $M$ is  $(\frac{1}{p})^{\frac{1}{p^n-1}}$.
\end{proposition}

\begin{proof}
Above.
\end{proof}

Taking $n=1$, we see that this agrees with the valuative capacity of $\mathbb{Z}$ computed in the second chapter. \\

What about $(\mathbb{Z}_{p_1} \times \mathbb{Z}_{p_2})$ for distinct primes? These spaces do not admit a scaling property, so the same toolset is not available. They are however semi-regular, so we know that\\  \[v_{\gamma_k}(\sigma(n)) =  \lfloor\frac{n}{\beta(k)}\rfloor - \lfloor\frac{n}{\beta(k+1)}\rfloor = \sum_{j=1}^{\alpha(k)-1} \lfloor \frac{n + j\cdot \beta(k)}{\alpha(k)\beta(k)} \rfloor \]
Suppose $p_1 =2$ and $p_2 =3$. Recall that the $\alpha$ sequence of $S=(\mathbb{Z}_{2} \times \mathbb{Z}_{3})$ counts the number of closed balls of radius $\gamma_{k+1}$ partitioning a closed ball of radius $\gamma_k$. In this case, $\Gamma_S$ is the non-positive powers of $2$ or $3$ sorted into decreasing order, so that $\Gamma_S$ starts $\{1, \frac{1}{2},\frac{1}{3},\frac{1}{4},\frac{1}{8},\frac{1}{9},\ldots \}$ and $\alpha(S)$ starts $\{6,2,3,2,2,3,2,3,2,\ldots\}$. The $\beta$ sequence of $S$, which counts the number of distinct balls of a fixed radius, then starts $\{6,12,36,72,144,\ldots\}$.\\

We know that the capacity of $S$ will be a product of some negative power of $2$ and some negative power of $3$.  From Lemma \ref{semi-regular formula}, we know that when $\alpha(k)=2$, we have\\ 
\[v_{\gamma_k}(\delta(n)) = \lfloor \frac{n + \beta(k)}{2\cdot \beta(k)} \rfloor\]

and when $\alpha(k)=3$, we have\\
\[v_{\gamma_k}(\delta(n)) = \lfloor \frac{n + \beta(k)}{3\cdot \beta(k)} \rfloor + \lfloor \frac{n + 2\cdot \beta(k)}{3\cdot \beta(k)} \rfloor \]

We also know that if $\alpha(k)=2$, then $\gamma_k$ must be a (negative) power of $2$, and likewise if $\alpha(k)=3$, then $\gamma_k$ is a power of $3$.\\

 Let us first explore the exponent of $2$ in $\delta(n)$. We start by noting that if $\gamma_k$ is some $2^{-i}$, then \[v_{\gamma_k}(\delta(n)) =\lfloor \frac{n + 2^i\cdot 3 ^j}{2^{i+1} \cdot 3^j} \rfloor \] since there will be a copy of $2$ in $\beta(k)$ for every occurence of $2$ in $\alpha(0),\ldots,\alpha(k)$, which is also what $i$ counts. So then, the exponent of $\frac{1}{2}$ in the $n^{th}$ characteristic sequence of $S$ is \[ \sum_{i=1}^\infty i \cdot \lfloor\frac{n + 2^i \cdot 3^j}{2^{i+1}\cdot 3^j} \rfloor\] What can we say about $j$, the exponent of $3$?\\ 

\begin{lemma}
Let $S = (\mathbb{Z}_{2} \times \mathbb{Z}_{3}) $ and consider the $k^{th}$ element of the $\beta$ sequence of $S$, $\beta(k) = 2^i \cdot 3^j$. If $k$ is such that $\gamma_k=2^{-i}$ for some $i$, then $j$ counts the numbers $a \in \mathbb{Z}_{\geq 0}$ such that $3^a < 2^{i}$.
\end{lemma}

\begin{proof}
$\Gamma_S$ is strictly monotone decreasing and each $\gamma_k$ is equal to a non-positive power of $2$ or $3$. If $\gamma_k = 2^i$, then all non-positive powers of $3$ and $2$ which are greater than $2^i$ must be equal to some $\gamma_j$, $0 \leq j < k$. That is, $2^i$ only appears in the $\Gamma_S$ sequence after all larger powers of $2$ and $3$ have been exhausted. Since we are only considering the case $\gamma_k$ is a power of $2$, this includes all of the smaller powers of $3$.
\end{proof}

 Now note that\\ 
\begin{align*}
3^a < 2^i
 \iff
 log_2(3^a) < log_2(2^i)
\iff  
a \cdot log_2(3) < i
\end{align*}

So now we are reduced to counting the number of non-negative integers $a$ that satisfy the above for a given $i$.  The number of such $a$'s will simply be the the value of the largest $a$ plus $1$ since $a$ satisfying the relation implies all $0 \leq a' \leq a$ solve the relation. Then, we are in fact reduced to finding the largest $a \in \mathbb{Z}$ that satisfies $a < \frac{i}{log_2(3)}$, but this is exactly $\lfloor \frac{i}{log_2(3)}\rfloor$. This in turn gives $j =  \lfloor \frac{i}{log_2(3)}\rfloor + 1= \lceil \frac{i}{log_2(3)}\rceil$, since $\frac{i}{log_2(3)}$ is never an integer. We now revisit our expression for the exponent of $\frac{1}{2}$ and substitute our new found value for $j$:\\

\begin{align} 
\sum_{i=1}^\infty i \cdot \lfloor\frac{n + 2^i \cdot 3^{\lceil \frac{i}{log_2(3)}\rceil}}{2^{i+1}\cdot 3^{\lceil \frac{i}{log_2(3)}\rceil}} \rfloor
=\sum_{i=1}^\infty i \cdot (\lfloor\frac{n}{2^i \cdot 3^{\lceil \frac{i}{log_2(3)}\rceil }}\rfloor -  \lfloor\frac{n}{2^{i+1}\cdot 3^{\lceil \frac{i}{log_2(3)}\rceil}} \rfloor)
\end{align}
% for n from 1 to 100 do
% evalf(Sum(x*floor((n*10000+2^x*3^ceil(x/log[2](3)))/(2^(x+1)*3^ceil(x/log[2](3)))), x = 1 .. infinity));
%end do;

%for n from 1 to 100 do
 %evalf(1/(n*100000)*Sum(x*floor((n*100000+2^x*3^ceil(x/log[2](3)))/(2^(x+1)*3^ceil(x/log[2](3)))), x = 1 .. infinity));
%end do;


%First, recall that in computing the valuative capacity of these spaces, we were ultimately reduced to finding solutions to polynomials of the form $x^{p^n} - \frac{x}{p}$ for some $n$ and for some $p$. The first observation is that these polynomials are $\mathbb{Z}$-valued on $p\mathbb{Z}$, that is, they are elements of $Int(p\mathbb{Z},\mathbb{Z})$. We might ask then, what sort of polynomials would arise in finding the valuative capacity of spaces such as $(\mathbb{Z}_2 \times \mathbb{Z}_3, \rho_\infty)$ or in computing the valuative capacity of infinite product spaces, such as $\mathbb{Z}_p \times \mathbb{Z}_p \times \mathbb{Z}_p \times \ldots$ for either some fixed prime $p$ or over each prime. \\

A symmetric argument shows that exponent of $\frac{1}{3}$ in the $i^{th}$ element of the $\rho_\infty-$sequence of $S$ is\\ 

\begin{align} 
\sum_{i=1}^\infty i \cdot \lfloor\frac{n + 2^{\lceil \frac{i}{log_3(2)}\rceil} \cdot 3^i}{2^{\lceil \frac{i}{log_3(2)}\rceil}\cdot 3^{i+1}} \rfloor
=\sum_{i=1}^\infty i \cdot (\lfloor\frac{n}{2^{\lceil \frac{i}{log_3(2)}\rceil } \cdot 3^i}\rfloor -  \lfloor\frac{n}{2^{\lceil \frac{i}{log_3(2)}\rceil}\cdot 3^{i+1}} \rfloor)
\end{align}

The sums that appear in $(5.1)$ and $(5.2)$ are real numbers that we, at present, know  little about. However, the aperiodicity of the sequences $\lceil\frac{i}{log_2(3)}\rceil$ and $\lceil\frac{i}{log_3(2)}\rceil$ over $i$ leads us to believe, but not prove, that each of the sums are irrational. We have the following conjecture.\\

\begin{conjecture} Finite products of $(\mathbb{Z}, \rho_{p_i})$ for distinct primes, $p_i$, have transcendental valuative capacity. \end{conjecture}
 
We end this section with an observation on the asymptotic behavior of capacity in these spaces. For a fixed prime $p$,  $(\frac{1}{p})^{\frac{1}{p^n-1}}$ is an monotone, increasing sequence in $n$ with $ lim_{n\to\infty} (\frac{1}{p})^{\frac{1}{p^n-1}} =  1$. For fixed $n$, the sequence in $p$ is also montone, increasing, again with  $ lim_{p\to\infty} (\frac{1}{p})^{\frac{1}{p^n-1}}=1$. In both cases, the limiting value is equal to the diameter of space. Indeed, we can observe that the sequence $\{(0,0,\ldots), (1,0,\ldots), (0,1,\ldots), \ldots\}$, in which the first element has only zeros and the $n$-th element has a single $1$ in the $(n-1)$-th component, is a $\rho-$ordering for both  $(\mathbb{Z}_p \times \mathbb{Z}_p \times \ldots, \rho_{p,\infty})$  and  $(\mathbb{Z}_2 \times \mathbb{Z}_3 \times \ldots, \rho_{P,\infty})$, since the distance between elements in this sequence (in either metric space) is always $1$. 

In considering the product space of ultrametric spaces, we may wonder whether the chosen metric also gives back the product topology on the space.  For products formed by taking some finite number of copies, the answer is positive. We give the necessary background and show this fact, adapting the proof in Munkres (20.3) to the case of ultrametric spaces. 

Under the $\rho_\infty$ metric these spaces are not compact. This situation is analogous to products of $\mathbb{R}$ and we can find an compact infinite product space but there is nothing canonical about it. 


We are now naturally left to ask whether the product topology on \textit{infinite} products of ultrametric spaces coincides with the $L_\infty$ metric. In this case, as in the analogous case of infinite copies of $\mathbb{R}$ and a uniform metric, the answer is negative (at least in general). Forunately, the metric that realizes the product topology on infinite copies of $\mathbb{R}$ can be adapted to the case of ultrametric spaces. We adapt to the proof of Munkres (20.5) to the case of infinite products of ultrametric spaces.

An important consequence of the fact that $d$ achieves the product topology is that Tychnoff's theorem then guarantees that product spaces formed with this metric will be compact, infinite or otherwise.  We consider two examples. 

\begin{example}
	Let $(\mathbb{Z}_p \times \mathbb{Z}_p \times \mathbb{Z}_p \times \ldots, d)$ be the metric space formed by taking the product of $(\mathbb{Z}_p, \rho_p)$ for some fixed prime $p$ and let $d$ be the product metric.
\end{example}

\begin{example}
	Let $(\mathbb{Z}_2 \times \mathbb{Z}_3 \times \mathbb{Z}_5 \times \ldots, \rho_{P,\infty})$ be the metric space formed by taking the product of $(\mathbb{Z}_p, \rho_p)$ for every prime $p$ and let $d$ be the product metric.
\end{example}




