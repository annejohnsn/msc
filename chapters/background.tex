\section*{Ultrametric basics}


\begin{definition*}
	 Let $(M, \rho)$ be a metric space. If $\rho$ satistifies the ultrametric inequality
	\[\rho(x,z) \leq max{(\rho(x,y), \rho(y,z))}, \forall x,y,z \in M\] 
	then (M, $\rho$) is an \textbf{ultrametric space}.
\end{definition*}

\begin{definition*}
	 Let $(V, N)$ be a normed vector space. Then $N$ satisfies the \textbf{strong trianlge inequality} if
	\[N(x + y) \leq max(N(x), N(y)), \forall x,y \in V \]
\end{definition*}

\begin{proposition*}
	Let $(V,N)$ be a normed vector space and suppose $N$ satisfies the strong triangle inequality. Then the metric space, $(V,\rho_N)$, where $\rho_N$ is the metric induced by $N$, is an ultrametric space.
\end{proposition*}

\begin{proposition*}
	\cite{ar} All triangles in an ultrametric space $(M,\rho)$ are either equilateral or isocoles, with at most one short side. 
\end{proposition*}


\begin{proposition*}
\cite{ar} If $S$ is a compact subset of an ultrametric space and $\Gamma_S$ is the set of all distances occurring between points of $S$, then $\Gamma_S$ is a discrete subset of $\mathbb{R}$. In particular if $\mid \Gamma_S\mid = \infty$, then the elements of $\Gamma_S$ can be indexed by $\mathbb{N}$.
\end{proposition*}

\noindent Let $(M, \rho)$ be a compact ultrametric space and let \[B_r(a)=\{x \in M \mid \rho(x,a) < r\}\] denote the open ball of radius $r$, centred at $a$ for some $r \in \mathbb{R}_{\geq 0}$ and $a \in (M,\rho)$. Likewise let \[\overline{ B_r(a)}=\{x \in M \mid \rho(x,a) \leq r\}\]  denote the closed ball of radius $r$, centred at $a$ for some $r \in \mathbb{R}_{\geq 0}$ and $a \in (M,\rho)$.

\begin{proposition*}
	Let $B_r(a)$ be a ball in an ultrametric space $(M,\rho)$. Then the diameter of $B$, $d=diam(B)=\sup_{x,y \in B}{\rho(x,y)}$, is less than or equal to the radius of $B$.    
\end{proposition*}

\begin{proposition*}
	If $(M, \rho)$ is an ultrametric space and $B_{r_1}(x_0)$ and $B_{r_2}(y_0)$ are balls in $(M, \rho)$, then either $B_{r_1}(x_0) \cap B_{r_2}(y_0) = \emptyset$, $B_{r_1}(x_0) \subseteq B_{r_2}(y_0)$, or $B_{r_2}(x_0) \subseteq B_{r_1}(x_0)$. That is, in an ultrametric space, all balls are either comparable or disjoint.
\end{proposition*}

\begin{proposition*}
\cite{ar} The distance between any two balls in an ultrametric is constant. That is, if $B_{r_1}(x_0)$ and $B_{r_2}(y_0)$ are two balls in an ultrametric space $(M,\rho)$, then $\rho(x,y)=c$ for some $c \in \mathbb{R}$ and $\forall x \in B_{r_1}(x_0)$ and $\forall y \in B_{r_2}(y_0)$
\end{proposition*}

\begin{proposition*}
\cite{ar} Every point of a ball in an ultrametric is at its centre. That is, if $B_r(x_0)$ is a ball in an ultrametric space $(M,\rho)$, then $B_r(x)=B_r(x_0)$,  $\forall x \in B_r(x_0)$
\end{proposition*}



