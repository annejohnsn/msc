\section{Ultrametric basics}


\begin{definition*}
	 Let $(M, \rho)$ be a metric space. If $\rho$ satistifies the ultrametric inequality
	\[\rho(x,z) \leq max{(\rho(x,y), \rho(y,z))}, \forall x,y,z \in M\] 
	then (M, $\rho$) is an \textbf{ultrametric space}.
\end{definition*}

\begin{definition*}
	 Let $(V, N)$ be a normed vector space. Then $N$ satisfies the \textbf{strong trianlge inequality} if
	\[N(x + y) \leq max(N(x), N(y)), \forall x,y \in V \]
\end{definition*}

\begin{proposition*}
	Let $(V,N)$ be a normed vector space and suppose $N$ satisfies the strong triangle inequality. Then the metric space, $(V,\rho_N)$, where $\rho_N$ is the metric induced by $N$, is an ultrametric space.
\end{proposition*}

\begin{proposition*}
	\cite{ar} All triangles in an ultrametric space $(M,\rho)$ are either equilateral or isocoles, with at most one short side. 
\end{proposition*}


\begin{proposition*}
\cite{ar} If $S$ is a compact subset of an ultrametric space and $\Gamma_S$ is the set of all distances occurring between points of $S$, then $\Gamma_S$ is a discrete subset of $\mathbb{R}$. In particular if $\mid \Gamma_S\mid = \infty$, then the elements of $\Gamma_S$ can be indexed by $\mathbb{N}$.
\end{proposition*}

\noindent Let $(M, \rho)$ be a compact ultrametric space and let \[B_r(a)=\{x \in M \mid \rho(x,a) < r\}\] denote the open ball of radius $r$, centred at $a$ for some $r \in \mathbb{R}_{\geq 0}$ and $a \in (M,\rho)$. Likewise let \[\overline{ B_r(a)}=\{x \in M \mid \rho(x,a) \leq r\}\]  denote the closed ball of radius $r$, centred at $a$ for some $r \in \mathbb{R}_{\geq 0}$ and $a \in (M,\rho)$.

\begin{proposition*}
	Let $B_r(a)$ be a ball in an ultrametric space $(M,\rho)$. Then the diameter of $B$, $d=diam(B)=\sup_{x,y \in B}{\rho(x,y)}$, is less than or equal to the radius of $B$.    
\end{proposition*}

\begin{proposition*}
	If $(M, \rho)$ is an ultrametric space and $B_{r_1}(x_0)$ and $B_{r_2}(y_0)$ are balls in $(M, \rho)$, then either $B_{r_1}(x_0) \cap B_{r_2}(y_0) = \emptyset$, $B_{r_1}(x_0) \subseteq B_{r_2}(y_0)$, or $B_{r_2}(x_0) \subseteq B_{r_1}(x_0)$. That is, in an ultrametric space, all balls are either comparable or disjoint.
\end{proposition*}

\begin{proposition*}
\cite{ar} The distance between any two balls in an ultrametric is constant. That is, if $B_{r_1}(x_0)$ and $B_{r_2}(y_0)$ are two balls in an ultrametric space $(M,\rho)$, then $\rho(x,y)=c$ for some $c \in \mathbb{R}$ and $\forall x \in B_{r_1}(x_0)$ and $\forall y \in B_{r_2}(y_0)$
\end{proposition*}

\begin{proposition*}
\cite{ar} Every point of a ball in an ultrametric is at its centre. That is, if $B_r(x_0)$ is a ball in an ultrametric space $(M,\rho)$, then $B_r(x)=B_r(x_0)$,  $\forall x \in B_r(x_0)$
\end{proposition*}

\section{$\rho$-orderings, $\rho$-sequences, and valuative capacity}

In what follows let $S$ be a compact subset of an ultrametric space $(M,\rho)$.

\begin{definition*}
\cite{kj} A \textbf{$\rho$-ordering} of $S$ is a sequence $\{a_i\}_{i=0}^\infty \subseteq S$ such that $\forall n > 0$, $a_n$ maximizes $\prod_{i=0}^{n-1} \rho(s,a_i)$ over $s \in S$. 
\end{definition*}

\begin{definition*}
\cite{kj} The \textbf{$\rho$-sequence} of $S$ is the sequence whose $0^{th}$-term is $1$ and whose $n^{th}$ term, for $n >0$, is $\prod_{i=o}^{n-1} \rho(a_n,a_i)$.
\end{definition*}

\begin{proposition*}
\cite{kj} The $\rho$-sequence of $S$ is well-defined so long as $S$ is compact and $\rho$ is an ultrametric. That is, the $\rho$-sequence of a compact subset of an ultrametric spaces does not depend on the choice of $\rho$-ordering of $S$.
\end{proposition*}

\begin{definition*}
\cite{kj}  Let $\gamma(n)$ be the $\rho$-sequence of $S$. The \textbf{valuative capacity} of $S$ is \[\omega(S)
 := \lim_{n\to\infty} \gamma(n)^{1/n}\]  
\end{definition*}


\begin{proposition*}
\cite{kj} For $S$ and $\gamma(n)$ as above,  $\lim_{n\to\infty} \gamma(n)^{1/n} = r < \infty$. 
\end{proposition*}


\begin{proposition*}
 If $S \subseteq M$ is a finite subset of an ultrametric space, then $\omega(S) =0$.
\end{proposition*}


\begin{proposition*}
(upper bound) If $diam(S)  := \max_{x,y \in S} \rho(x,y)= d$, then $\omega(S) < d$.
\end{proposition*}

\begin{proof}
Since $d$ is the diameter of $S$, the $n^{th}$ term of the $\rho$-sequence of $S$ is bounded by $d^n$ and so $ \lim_{n\to\infty} \gamma(n)^{1/n}=d$ if and only if $\gamma(n)=d^n$, $\forall n$. This implies $\rho(a_n, a_i) = d$, $\forall n$ and $\forall i < n$, but then $\rho(a_i,a_j)=d$, $\forall i,j$, since the $\rho$-sequence is maximized at each $n$. This means $\omega(S) < d$ would imply that the cover of $S$, $\cup_{a_i} B_d(a_i)$ is in fact an infinite partition, contradicting the compactness of $S$. Then  $\omega(S)= \lim_{n\to\infty} \gamma(n)^{1/n}<d$. 
\end{proof}



\begin{proposition*}
	(translation invariance) Let $(M, \rho)$ be a compact ultrametric space and suppose $M$ is also a topological group. If $\rho$ is (left) invariant under the group operation, then so is $\omega$. That\ is, if $\rho(x,y)=\rho(gx,gy)$, $ \forall g,x,y \in M$, then $\omega(gS)=\omega(S)$, for $S \subseteq M$.	
\end{proposition*}

\begin{proof}
Let $\{a_i\}_{i=0}^\infty$ be a $\rho$-ordering for $S$. Then $\{ga_i\}_{i=0}^\infty$ is a $\rho$-ordering for $gS$. Then $$\omega(gS) = \lim_{n\to\infty} \gamma(n)^{1/n} =  \lim_{n\to\infty} [\prod_{i=0}^{n-1} \rho(ga_n,ga_i)]^{1/n} = \lim_{n\to\infty} [\prod_{i=0}^{n-1} \rho(a_n,a_i)]^{1/n}	 = \omega(S)$$
\end{proof}	

\begin{example}
With the notation of the previous section, note that for $x,y \in (\mathbb{Z}_p, \mid \cdot \mid_p)$, $\rho_p(x,y) = \mid x - y \mid_p = p^{-\nu_p(x-y)} = p^{-\nu_p((a+x)-(a+y))} =  \mid (a+x) - (a+y) \mid_p = \rho_p(a+x,a+y)$ so that $\omega(a+S) = \omega(S)$ for $S \in (\mathbb{Z}_p, \mid \cdot \mid_p)$.
\end{example}

\begin{example}
Let $(\mathbb{Z}_p \times \mathbb{Z}_p, \rho_{p,\infty})$ be the metric space with elements $\{(x,y)\mid x,y \in \mathbb{Z}_p\}$ and metric $\rho_{p,\infty}((x_1,x_2), (y_1,y_2)) = \max(\rho_p(x_1, y_1)), \rho_p(x_2, y_2))$. Consider it also as a topological group with operation $(g_1,g_2) + (x_1,x_2) = (g_1+x_1, g_2+x_2)$. Then $\rho_{p,\infty}((x_1,x_2), (y_1,y_2))=\max(\rho_p(x_1, y_1)), \rho_p(x_2, y_2)=\max(\rho_p(g_1+x_1, g_1+y_1)), \rho_p(g_2+x_2, g_2+y_2)=\rho_{p,\infty}(((g_1,g_2) + (x_1,x_2)), ((g_1,g_2) + (y_1,y_2)))$, and $\omega((g_1,g_2)+S) = \omega(S)$ for $S \in (\mathbb{Z}_p \times \mathbb{Z}_p, \rho_{p,\infty})$. 	
	
\end{example}
\begin{proposition*}
Let $(V, N)$ be a normed vector space and suppose $N$ satisfies the strong triangle identity. Then if $N$ is multiplicative, so is $\omega$. That is, if $N(gx)=N(g)N(x)$,$\forall g,x \in V$, then $\omega(gS) = N(g)  \omega(S)$, for $g \in V$ and $S \subseteq M$. 
\end{proposition*}

\begin{proof}
Let $\rho$ be the metric induced by $N$, so that $\rho(x,y) = N(x-y), \forall x,y \in V$. Let $\{a_i\}_{i=0}^\infty$ be a $\rho$-ordering for $S$. Then since $N$ is multiplicative, for $u, v \in gS$, $u=gs_i$ and $v=gs_j$ for some $s_i, s_j \in S$,  $$\rho(u, v) = \rho(gs_i, gs_j) =N(gs_i - gs_j) = N(g(s_i - s_j)) = N(g)N(s_i - s_j) = N(g)\rho(s_i,s_j).$$
Then $\{ga_i\}_{i=0}^\infty$ is a $\rho$-ordering for $gS$ and 

 $$\omega(gS) = \lim_{n\to\infty} [\prod_{i=0}^{n-1} \rho(ga_n,ga_i)]^{1/n} 
 = \lim_{n\to\infty} [\prod_{i=0}^{n-1} N(g)\rho(a_n,a_i)]^{1/n} $$
 $$= \lim_{n\to\infty} [N(g)^n\prod_{i=0}^{n-1} \rho(a_n,a_i)]^{1/n} = N(g) \lim_{n\to\infty} [\prod_{i=0}^{n-1} \rho(a_n,a_i)]^{1/n} = N(g) \omega(S)$$
\end{proof}


\begin{example}
	Since $\mid \cdot \mid_p$ is multiplicative, $\omega(mS) = \mid m \mid_p  \omega(S)$ for $m \in \mathbb{Z}_p$ and $S \subseteq \mathbb{Z}$. In particular, $\omega(p\mathbb{Z}) = \mid p \mid_p \omega(\mathbb{Z}) = \frac{1}{p}*p^{\frac{1}{1-p}} = p^{-p/p-1}.$
\end{example}

\begin{example}
	Let $(\mathbb{Z}_p \times \mathbb{Z}_p, \mid \cdot \mid_{p,\infty})$ be the vector space with elements $\{(x,y)\mid x,y \in \mathbb{Z}_p\}$ and norm $ \mid (x_1,x_2) \mid_{p,\infty} = \max(\mid x_1 \mid_p,\mid x_2 \mid_p ).$ Then $\mid (g,g)(x_1,x_2) \mid_{p,\infty} = \max(\mid gx_1 \mid_p,\mid gx_2 \mid_p ) = \max(\mid g \mid_p \mid x_1 \mid_p,\mid g \mid_p \mid x_2 \mid_p=\mid (g,g) \mid_p \mid (x_1,x_2) \mid_p$, so that $\mid \cdot \mid_{p,\infty} $ is multiplicative for $(g,g) \in \mathbb{Z}_p \times \mathbb{Z}_p$. Then $\omega((g,g)S) = \mid (g,g) \mid_p\omega(S)$. In particular, $\omega((p,p)\mathbb{Z} \times \mathbb{Z}) =\omega(p\mathbb{Z} \times p\mathbb{Z})=\mid (p,p) \mid_p \omega(\mathbb{Z} \times \mathbb{Z}) =
	p^{-1} \omega(\mathbb{Z} \times \mathbb{Z}). $
\end{example}


\begin{proposition*}
	\cite{kj}(subadditivity) If  $diam(S) := \max_{x,y \in S} \rho(x,y)=d$ and $S=\cup_i^n A_i$ for $A_i$ compact subsets of $M$ with $\rho(A_i, A_j)=d, \forall i,j$, then \[\frac{1}{log(\omega(S)/d) } = \sum_{i=1}^n \frac{1}{log(\omega(A_i)/d)}\] 
\end{proposition*}

\begin{corollary*}
	Suppose $S = \cup_i^n S_i$ with $\rho(S_i, S_j)=d=diam(S)$ and also $\omega(S_i)=\omega(S_j)$, $\forall i,j$ .  Let $r \in \mathbb{R}$ be such that $\omega(S_i)=r\omega(S)$, $\forall i$. Then $\omega(S) = r^{\frac{1}{n-1}}\cdot d$. In particular if $S = \mathbb{Z}$ and $(M,\rho)= (\mathbb{Z}, \mid \cdot\mid_p)$ then $\omega(S)=(\frac{1}{p})^{1/p-1}$ for any prime $p$. 
\end{corollary*}

\begin{corollary*}
	(Joins of computable sets are computable) Let  $\Gamma_M = \{\gamma_0, \gamma_1,\ldots, \gamma_\infty=0\}$ be the set of distances in $M$. Suppose that $S = B_{\gamma_i}(x)$,  for some $x$ and $i$, is the union of $2$ or more balls of radius $\gamma_{i+1}$, i.e., $S=\cup_{j=1}^n B_{\gamma_{i+1}} (x_j)$ is a join in the lattice of open sets in $M$, then 
	\[\frac{1}{log(\omega(S)/\gamma_{i+1} )} = \sum_{j=1}^n \frac{1}{log(\omega(B_{\gamma_{i+1}}(x_j))/\gamma_{i+1} )}\]
\end{corollary*}

\section{Background from Gerritzen and van der Put}

Background results from \cite{gvdp}. Let $k$ be a field that is complete with respect to a non-archimedean valuation and let $K$ be a complete and algebraically closed field containing $k$. 
\begin{definition*}\cite{gvdp}
	The set $\{\lambda \in k; \mid \lambda\mid \leq 1\}$, denoted $k^0$, is the \textbf{valuation ring} of $k$. It has a unique maximal ideal, denoted $k^{00}$, given by $\{\lambda \in k; \mid \lambda\mid < 1\}$. The \textbf{residue field} of $k$ is $\bar{k} := k^0/k^{00}$.
\end{definition*}


\begin{definition*}\cite{gvdp} The \textbf{projective line over $k$}, denoted $\mathbb{P}^1(k)$, is the space whose points are lines $l$ in $k^2$ that intersect $(0,0)$ and whose topology and field structure are inherited from $k$. 
\end{definition*}

We give two equivalent representations for the points of $\mathbb{P}^1(k)$. A point $p \in \mathbb{P}^1(k)$ is an equivalence class of $k^2 \setminus (0,0)$ under the relation $(x,y) \sim (x',y')$ if there exists a $\lambda \in k\setminus 0$ such that  $(x,y) = \lambda(x',y')$. Equivalently, suppose that $l$ is a line in $k^2$ intersecting the origin, that is a point in $\mathbb{P}^1(k)$. We denote $l$ by a representative $[x_0, x_1] \in k^2$ such that $l = \{\lambda (x_0, x_1 ); \lambda \in k\}$, called homogeneous coordinates of $l$.



\begin{proposition*} \cite{gvdp} Let $\psi: k \rightarrow \mathbb{P}^1(k)$ be the map given by $\psi(\lambda_0) = [1, \lambda_0]$, where $ [1, \lambda_0]$ is the line in $k^2$, $\{\lambda(1, \lambda_0); \lambda \in k\}$. Then the image of $\psi$ is $\mathbb{P}^1(k) \setminus [0,1]$ and is isomorphic to $k$, so that $k$ is identified with projective space minus a distinguished point, $[0,1]$, which is denoted by $\infty$.  
\end{proposition*}

\begin{definition*}\cite{gvdp} $k$ is called a \textbf{local field} if $k$ is locally compact. \end{definition*}

\begin{proposition*} \cite{gvdp} The following are equivalent:
	\begin{enumerate}
		\item $k$ is a local field.
		\item $\mid k^*\mid \cong \mathbb{Z}$ and $\bar{k}$ is finite, where $k*$ is the set of units in $k$, ie $k^* = \{\lambda \in k, \lambda \neq 0\}$.
		\item $k$ is a finite extension of either $\mathbb{Q}_p$ or $\mathbb{F}_p((t))$.
		\item $\mathbb{P}^1(k)$ is compact
	\end{enumerate}
\end{proposition*}	

\begin{definition*} \cite{gvdp}
We denote by $GL(2,k)$ the set of invertible $2 \times 2$ matrices over $k$. A \textbf{fractional linear automorphism}, $\phi$, of $\mathbb{P}^1(k)$ is a map  defined by $z \mapsto \frac{az +b}{cz +d}$ for some 
$\bigl( \begin{smallmatrix}a & b\\ c &d\end{smallmatrix}\bigr) \in GL(2,k)$. The set of fractional linear automorphisms of $\mathbb{P}^1(k)$ is denoted $PGL(2,k)$. Note that $PGL(2,k) = GL(2,k) / \{ \bigl( \begin{smallmatrix}\lambda & 0\\ 0 &\lambda \end{smallmatrix}\bigr); \lambda \in k^*  \}$. In homogeneous coordinates, we can represent the action of $\phi$ by $[x_0,x_1] \mapsto [cx_1 +dx_0, ax_1 +bx_0]$. 
\end{definition*}



\begin{definition*} \cite{gvdp}
Suppose $\Gamma$ is a subgroup of $PGL(2,k)$. A point $p  \in \mathbb{P}^1(k)$ is a \textbf{limit point of $\Gamma$}, if there exists a point $q$ in  $\mathbb{P}^1(k)$ and a sequence $\{\gamma_n\}_{n\geq 1}$ in $\Gamma$ such that $\lim_{n\to\infty} \gamma_n(q) = p$.
\end{definition*}

\begin{proposition*} \cite{gvdp} If $\Gamma$ is not a discrete subgroup of $PGL(2,k)$ then every point of  $\mathbb{P}^1(k)$ is a limit point of $\Gamma$.
\end{proposition*}	

\begin{proof} 
Since $\Gamma$ is not discrete, the sequence  $\{\gamma_n\}_{n\geq 1}$ has a limit $\gamma$ in $\Gamma$. Let $p$ be any point of $\mathbb{P}^1(k)$ and let $q= \gamma^{-1}(p)$. Then $\lim_{n\to\infty} \gamma_n(q) = \lim_{n\to\infty} \gamma_n(\gamma^{-1}(p)) =p$. 
\end{proof}	

\begin{definition*} \cite{gvdp} A subgroup $\Gamma$ of $PGL(2,k)$ is \textbf{discontinuous} if the closure of every orbit of $\Gamma$ in $\mathbb{P}^1(k)$ is compact and the set of all limit points of $\Gamma$ is not equal to  $\mathbb{P}^1(k)$ .
\end{definition*}

\begin{proposition*} \cite{gvdp} If $\Gamma$ is a discontinuous subgroup of $PGL(2,k)$  and $\mathcal{L}$ is the set of limit points of $\Gamma$, then $\mathcal{L}$ is compact, no where dense and if $\mathcal{L}$ contains more than two points, $\mathcal{L}$ is perfect.
\end{proposition*}

\begin{definition} \cite{gvdp}
Let $A$ be an element of $GL(2,k)$ and let $a_1$ and $a_2$  be the eigenvalues of $A$. Then $A$ is called \textbf{elliptic} if $a_1 \neq a_2$, but $\mid a_1\mid = \mid a_2\mid$. $A$ is called \textbf{parabolic} if $a_1 = a_2$, and $A$ is called \textbf{hyperbolic} if  $\mid a_1\mid \neq \mid a_2\mid$.
\end{definition}

\begin{example} 
Consider the matrix $T_s =  \bigl( \begin{smallmatrix}p & s\\ 0 & 1 \end{smallmatrix}\bigr) \in GL(2, \mathbb{Q}_p)$ for some $s$ in $(0,\ldots, p-1)$. $T_s$ has eigenvalues $p$ and $1$ and so $T_s$ is hyperbolic for any choice of $s$ or $p$. Consider the action of $T_s$ on $\mathbb{Z}_p \subset \mathbb{Q}_p$, where $\mathbb{Z}_p$ is identified with the subspace $\{[1,\lambda];\lambda \in \mathbb{Z}_p \}$ of $\mathbb{P}^1(\mathbb{Q}_p)$. In homogeneous coordinates, this action is given by $[1,\lambda] \mapsto [1, p\lambda +s]$. Since $\mid (p\lambda +s -s)\mid = \mid p\lambda\mid \leq \frac{1}{p}$, $T_s$ sends $\lambda$ to $B_\frac{1}{p}(s)$. Also note that for $s = 0$, $T_s$ has the effect of shifting the index of $\lambda$ by 1, that is, if $\lambda = \sum_{i=n}^\infty a_ip^i$, where $n = ord(\lambda)$, then $T_0([1,\lambda]) = [1, p\lambda] \rightsquigarrow p\lambda = \sum_{i=n+1}^\infty a_{(i-1)}p^{(i-1)}$.	
\end{example}	

\begin{definition*} \cite{gvdp} A \textbf{Schottky group} is a finitely-generated, free and discontinuous subgroup of  $PGL(2,k)$
\end{definition*}


