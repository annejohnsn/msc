%\section{Ultrametric basics}

The principal context for this thesis is an arbitrary ultrametric space, which is a metric space that also satisifies an additional axion, sometimes called the ultrametric inequality or (in the case of vector spaces) the strong triangle propery. We define  ultrametric spaces below and for the rest of this section, we review some of their more important characteristics. The proofs offered in this section are, for the most part, standard and can be found in a number of reference texts, such as \cite{ar}.\\

\begin{definition}
	 Let $(M, \rho)$ be a metric space; that is, suppose $M$ is a set and $\rho: M \times M \rightarrow \mathbb{R}_{\geq 0}$ is such that:
	\begin{enumerate}[(i)]
		\item $\rho(x,y) = 0$ if and only if $x=y$
		\item $\rho(x,y) = \rho(y,x)$
		\item $\rho(x,z) \leq \rho(x,y) + \rho(y,z)$
	\end{enumerate}
	for any $x,y,z \in M$. If $\rho$ satistifies the ultrametric inequality,	
		\[ \rho(x,z) \leq \max{(\rho(x,y), \rho(y,z))}\]
for any $x,y,z \in M$, then $(M, \rho)$ is an \textbf{ultrametric space}.
\end{definition}

A special case of an ultrametric space, and one where much of the previous work on this topic has been completed, is one where the metric has been derived from a norm on a vector space. \\

\begin{definition}
	 Let $(V, N)$ be a normed vector space; that is, suppose $V$ is $\mathbb{F-}$vector space, for $\mathbb{F}$ some subfield of $\mathbb{C}$, and $N: V \rightarrow \mathbb{R}_{\geq 0}$  is such that:
	\begin{enumerate}[(i)]
		\item $N(x +y ) \leq N(x) + N(y)$
		\item $N(cx) = \lvert c \rvert \text{ } N(X)$
		\item $N(x) = 0$ implies $x=0$
	\end{enumerate}
	for any $x,y \in V$ and $c \in \mathbb{F}$. We say that $N$ satisfies the \textbf{strong trianlge inequality} if
	\[ N(x + y) \leq \max(N(x), N(y)) \]
	for any $x,y \in V$.
\end{definition}

\begin{proposition}
	Let $(V,N)$ be a normed vector space and suppose $N$ satisfies the strong triangle inequality. Then the metric space, $(V,\rho_N)$, where $\rho_N$ is the metric induced by $N$, that is, $\rho_N(x,y) = N(x-y)$, is an ultrametric space.
\end{proposition}

\begin{proof}
We take for granted that  $(V,\rho_N)$ is a metric space and also note that 
\[N(x + z) \leq \max(N(x), N(z))\]
 implies  
\[\rho_N(x,z) \leq \max(\rho_N(x,0), \rho_N(z,0)) \leq \max(\rho_N(x,y), \rho_N(y,z))\]
\end{proof}

\begin{notation*}
 If $(V, N)$ is a normed vector space, then the metric induced by $N$ is denoted $\rho_N$.
\end{notation*}

When ultrametric spaces come from spaces with algebraic structure, such as normed vector spaces, some of this structure carries over into metric spaces structure in a rather nice way:\\

\begin{proposition}
\cite{ar} Let $S$ be a  group equipped with a (right) invariant ultrametric, $\rho$. If $B=B(0,r)$ is a (closed) ball centred at the neutral element of $S$, then $B$ is a subgroup of $S$.
\end{proposition}

\begin{proof}
Let $x,y \in B$. Then \[\rho(x-y,0) = \rho(x,y)  \leq \max(\rho(x,0), \rho(y,0)) \leq r,\]
so that $x-y \in B$.
\end{proof}

The canoncial example of an ultrametric space, and one which is derived from a norm, is $p-$adic integers, $\mathbb{Z}_p$ for any prime $p$. 
\begin{definition}
$p-$adic valuation
\end{definition}

It is easy to see that the $p-$adic valuation exhibits the following properites, which makes it nice to work with:

\begin{itemize}
\item $v_p(ab) = v_p(a) + v_p(b)$
\item $another one$
\end{itemize}

\begin{example} $\mathbb{Z}_p$ \end{example}

We give a second example below, showing a space that does not readily come equipped with algebraic structure:

\begin{example} snowflake metric \end{example}

Ultrametric spaces exhibit properties much unlike traditional metric spaces, and we review of few of these below. Of particular interest to us is the behavior between (closed) balls in an ultrametric space.\\

\begin{notation*}
	Let $(M, \rho)$ be a compact ultrametric space and let \[B(a, r)=\{x \in M \mid \rho(x,a) \leq r\}\] denote the \textit{closed} ball of radius $r$, centred at $a$ for some $r \in \mathbb{R}_{> 0}$ and $a \in (M,\rho)$.
\end{notation*} 
In the above notation, we break from  convention in that we denote a closed ball without using any decoration and in that we omit any notation for an open ball. This because, for the most part, the notion of open and closed ball in an ultrametric space overlap, although we will need a few more facts before showing this.\\

\begin{definition}
Let $S$ be a subset of an ultrametric space. The \textbf{diameter of $S$} is $diam(S) = \sup\limits_{x,y\in S}\rho(x,y)$. Note that if $S$ is compact, $diam(S) = \max\limits_{x,y\in S}\rho(x,y)$.
\end{definition}

\begin{proposition}
	Let $B=B(a, r)$ be a (closed) ball in an ultrametric space $(M,\rho)$. Then the diameter of $B$ is less than or equal to the radius of $B$.    
\end{proposition}

\begin{proof}
Suppose $d = diam(B) > r$. This would imply there exists $x,y$ in $B$ such that $\rho(x,y) > r$, in particular $\rho(x,y)$ is strictly greater than $\max(\rho(x,a), \rho(y,a))$, which is a contradiction since $\rho$ is an ultrametric.
\end{proof}

In the following proposition, we describe the triangles in an ultrametric space, and the result is more or less a restatement, in geometric terms, of the ultrametric inequality.\\

\begin{proposition}
	 All triangles in an ultrametric space $(M,\rho)$ are either equilateral or isosceles, with at most one short side. 
\end{proposition}

\begin{proof}
Let $x,y$, and $z$ be three points in an ultrametric space $(M,\rho)$. We show that $\rho(x,y) \neq \rho(x,z)$ and $\rho(x,y) \neq \rho(y,z)$ implies $\rho(x,y) < \rho(x,z) = \rho(y,z)$.\\

If $\rho(x,z) \neq \rho(y,z)$, then without loss, $\rho(x,z) > \rho(y,z)$. At the same time, the ultrametric inequality implies $\rho(x,y) \leq \max(\rho(x,z), \rho(y,z))$ and  $\rho(y,z) \leq \max(\rho(x,y), \rho(x,z))$. The first inequality implies $\rho(x,y) < \rho(x,z)$, which means the second inequality implies $\rho(y,z) < \rho(x,z)$. This is a contradiction, so we must have $\rho(x,z) = \rho(y,z)$.\\

To see that $\rho(x,y) < \rho(x,z)$, simply note that $\rho(x,y) \leq \max(\rho(x,z), \rho(y,z))$
\end{proof}

With this result in hand, we are able to quickly demonstrate some of the properties of (closed) balls, which are of fundamental importance to us. We see below that the ultrametric inequality, perhaps innocuous on the surface, quickly implies ultrametric balls are markedly different from their archimedean counterparts.\\

\begin{proposition}
 Every point of a ball in an ultrametric is at its centre. That is, if $B(x_0, r)$ is a ball in an ultrametric space $(M,\rho)$, then $B(x, r)=B(x_0, r)$,  $\forall x \in B(x_0, r)$
\end{proposition}

\begin{proof}
Let $a \in B(x, r)$. Then $\rho(a,x) \leq r$ and since \[\rho(a,x_0) \leq \max(\rho(a,x), \rho(x,x_0)) \leq r\] we must have $a \in B(x_0, r)$ and  $B(x, r) \subseteq B(x_0, r)$ A similar argument shows $B(x_0, r) \subseteq B(x, r)$.
\end{proof}

\begin{proposition}
	If $(M, \rho)$ is an ultrametric space and $B(x_0, r_1)$ and $B(y_0, r_2)$ are balls in $(M, \rho)$, then either $B(x_0, r_1) \cap B(y_0, r_2) = \emptyset$, $B(x_0, r_1) \subseteq B(y_0, r_2)$, or $B(x_0, r_1) \subseteq B(x_0, r_1)$. That is, in an ultrametric space, all balls are either comparable or disjoint.
\end{proposition}

\begin{proof}
Suppose $B(x_0, r_1) \cap B(y_0, r_2) \neq \emptyset$ and let $z$ be a point in the intersection. We show that if there exists an $a \in B(y_0, r_2)$ such that $a \notin B(x_0,r_1)$, then $B(x_0, r_1) \subseteq B(y_0, r_2)$. Let $x \in B(x_0, r_1)$. Then we must have $\rho(x,z) < \rho(x,a)$, since $z \in B(x_0, r_1) = B(x,r_1)$ and $a$ is not. Since the triangle with vertices $(a,x,z)$ is isocolces with at most one short side, we must have $\rho(x,a) = \rho(a,z) \leq r_2$, since $a \in B(y_0, r_1) = B(z,r_2)$. Then $x \in B(y_0, r_1)$. 
\end{proof}

\begin{proposition}
 The distance between any two non-overlapping balls in an ultrametric is constant. That is, if $B(x_0, r_1)$ and $B(y_0, r_2)$ are two balls in an ultrametric space with $B(x_0, r_1) \cap B(y_0, r_2) = \emptyset$, then there exists a $c \in \mathbb{R}_{\geq 0}$ such that  $\rho(x,y)=c$, $\forall x \in B(x_0, r_1)$ and $\forall y \in B(y_0, r_2)$.
\end{proposition}

\begin{proof}
Suppose $\rho(x_0, y_0)=c$ and let $x \in B(x_0, r_1)$ and $y \in B(y_0, r_2)$ be arbitrary. Consider the triangle formed by $(x_0, y_0,y)$. Since $\rho(x_0,y_0)=c$ and  $\rho(y,y_0) \leq r_2 < c$, we must have $\rho(x_0, y)=c$ because triangles in an ultrametric space have at most one short side. Now consider the triangle formed by  $(x_0, x,y)$. Since $\rho(x_0, y)=c$ and  $\rho(x,x_0) \leq r_1 < c$, we must have $\rho(x, y)=c$.
\end{proof}


Before moving on, we give another example of an ultrametric space, showing again the quick and unusal implications of Proposition $4$.\\

\begin{example}
Let $S$ be a compact subset of an ultrametric space, $(M,\rho)$, and then let $\textbf{Triag}_S$ be the space whose points are the set of distinct (up to labeling) triangles in $S$. If $t$ is an element of $\textbf{Triag}_S$, denote by $long_t$ the length of the long side of $t$ and by $short_t$  the length of the short side of $t$. Then define $\Delta_S(t,t') = \lvert long_t - long_{t'}\rvert + \lvert short_t - short_{t'} \rvert$, for $t$ and $t' \in \textbf{Triag}_S$. Now note $long_t$ and $short_t$ are in $\mathbb{R}$ and so $\Delta_S(t,t')$ is always in $\mathbb{R}_{\geq 0}$. It's also clear that $\Delta_S(t,t') = 0$ if, and only if, $t$ and $t'$ have the same length sides, in which case they are equal in $\textbf{Triag}_S$. Now let $t_a, t_b$ and $t$ be any three points in $\textbf{Triag}_S$. Without loss of generality, assume $\max(\Delta_S(t_a,t), \Delta_S(t_b,t))=\Delta_S(t_b,t)$, that is,
\[ \lvert long_{t_a} - long_{t}\rvert + \lvert short_{t_a} - short_{t} \rvert \leq \lvert long_{t} - long_{t_b}\rvert + \lvert short_{t} - short_{t_b} \rvert \]
We must show
\[\lvert long_{t_a} - long_{t_b}\rvert + \lvert short_{t_a} - short_{t_b} \rvert \leq \lvert long_{t} - long_{t_b}\rvert + \lvert short_{t} - short_{t_b} \rvert\]
You can easily construct a counter example with all equilateral triangles (which abouts to the fact that $\mathbb{R}$ is not normally an ultrametric space)
\end{example}
Note that an analogous construction would not work on the set of line segments in $S$.\\

The following proposition is  easy to see, although the result is both unintuitive and important for our purposes. \\

\begin{proposition}
Suppose $S$ is a compact subset of an ultrametric space $(M, \rho)$ and that $\cup_{i \in I} B(x_i,r_i)$ is a cover of $S$ by (closed) balls in $S$. Then there exists $i_1,\ldots, i_n$, a finite subset of $I$, such that  $\cup_{j=1}^{j=n} B(x_{i_j},r_{i_j})$ is a partition of $S$.
\end{proposition}

\begin{proof}
Since $S$ is compact, $\cup_{i \in I} B(x_i,r_i)$ contains a finite subcover of $S$. Say this subcover is given by the elements  $i_1,\ldots, i_{n'} \in I$, and suppose this is not a partition. That is, suppose for some $i_i, i_j$, $ B(x_{i_i},r_{i_i}) \cap  B(x_{i_j},r_{i_j}) \neq \emptyset$. Then, without loss of generality, we must have $ B(x_{i_i},r_{i_i}) \subseteq  B(x_{i_j},r_{i_j})$, so that the removal of $ B(x_{i_i},r_{i_i})$ is still a cover of $S$. We continue this process a finite number of times, since the subcover was finite to begin with, to arrive at a partition of $S$.
\end{proof}

In fact, a slightly stronger statement then the above is true:

\begin{corollary}
Suppose $S$ is a compact subset of an ultrametric space $(M, \rho)$ and that $B(x_0,r)$ is a (closed) ball in $S$. Then, there exists a finite partition of $S$ having $B(x_0,r)$ as an element.
\end{corollary}

\begin{proof}
Let $\mathcal{C}$ be the cover of $S$ given by $\cup_{x\in S} B(x,r) \cap S$. From the proposition, we can select a finite subcover of $\mathcal{C}$ that is a partition of $S$. Suppose $B(y,r) \cap S$ is the element in this partition containing $x_0$. Then  since $B(y,r)$ and $B(x_0,r)$ are equal in $M$, $B(y,r) \cap S = B(x_0,r) \cap S = B(x_0,r)$.
\end{proof}

We end this section by making a few comments about the set of distances that occur between the points of a compact ultrametric space.\\

\begin{proposition}
\cite{ar} Let $S$ is a compact subset of an ultrametric space. For $a \in S$, let $\phi_a: S \setminus \{a\} \rightarrow \mathbb{R}$ be the function defined by $\phi_a(x)=\rho(x,a)$. Then $Im(\phi_a)$ is a discrete subset of $\mathbb{R}$ for all $a \in S$.
\end{proposition}

\begin{proof}
\cite{ar} %Recall that all triangles in an ultrametric space are isosceles with one short side. In particular, for $x,y,a \in S$, for $S$ some compact subset of an ultrametric space $(M,\rho)$, \[ \rho(x,y) < \rho(x,a) \text{ implies } \rho(y, a) = \rho(x,a)\] 
%so that if $r \in \mathbb{R}, 0 < r < \rho(x,a)$, then $\phi_a$ restricted to $B(x,r)$ is constant for any $a \in S$. That is, $\phi_a$ is locally constant. 
%The fibers, $\phi_a^{-1}(c) = \{x \in S \mid \rho(x,a) =c \}$ for $c \in Im(\phi_a)$, form an open partition of $S$.
\end{proof}


\begin{corollary}
\cite{ar} Let $B(a, r)$ be a closed ball in an ultrametric space. Then there exists $ r' > r \in \mathbb{R}$ such that $B(a, r) = \{x \in M \mid \rho(x,a) < r'\} $; that is, every closed ball is also an open ball with the same centre and  slightly larger radius.
\end{corollary}

\begin{proof}
\end{proof}

\begin{corollary}
\cite{ar} If $S$ is a compact subset of an ultrametric space and $\Gamma_S$ is the set of all distances occurring between points of $S$, then $\Gamma_S$ is countable; that is, there is an injective function from $\Gamma_S \mapsto \mathbb{N}$.
\end{corollary}

\begin{proof}
%Most recent working idea: create an ultrametric space, $\textbf{Triag}_S$, whose points are the set of distinct (up to labeling) triangles in $S$. Show that an infinite number of limit points in $\Gamma_S$ implies there exists some $t \in \textbf{Triag}_S$ such that $\phi_T$ is not discrete.
\end{proof}

It will become useful to write the set of distances occuring in $S$ as a sequence, put in decreasing order. 

\begin{notation*}
If $S$ is a compact (hence bounded) ultrametric space, then we denote the set of distances between points of $S$ by 
$$\Gamma_S = \{\gamma_0 = d =diam(S), \gamma_1, \gamma_2, \ldots, \gamma_\infty =0 \}$$

where $\gamma_i \in  \Gamma_S$ if and only if $\exists x,y \in S$ such that $\rho(x,y) = \gamma_i$ and  $\gamma_i < \gamma_j$ if and only if $i > j$. 
\end{notation*}


To make use of regularity in $S$ we should, before continuing, also put some constraints on our sequence of distances $\Gamma_S$.

\begin{definition}
Let $S$ be a compact subset of an ultrametric space and $\Gamma_S$ is the sequence of decreasing distances in $S$. Then we say $S$ is \textbf{well-behaved} (I guess - or tame? reasonable?), if $S$ is semi-regular and $\gamma_k = \alpha(k)^{c_k}$ for all $k \in \mathbb{N}$ and some $c_k \in \mathbb{Z}$.
\end{definition}

\begin{example}
Any ultrametric space where $\rho$ is induced from a valuation domain.
\end{example}



\newpage
\section*{$\rho$-orderings, $\rho$-sequences, and valuative capacity}

We now come to the central theme of this work. The observation that an analogous notion of $p-$ordering can be defined for a general ultrametric space, and that these structures coincide with Fekete $n-$tuples, is due to \cite{kj}. The exploration of this idea makes up the remainder of this work. \\ 

In what follows, we let $S$ be a compact subset of an ultrametric space $(M,\rho)$.\\

\begin{definition}
	\cite{kj} A \textbf{$\rho$-ordering} of $S$ is a sequence $\{a_i\}_{i=0}^\infty$ in $S$ such that $a_0$ is arbitrary and $\forall n > 0$, $a_n$ maximizes $\prod_{i=0}^{n-1} \rho(s,a_i)$ over $s \in S$. 
\end{definition}

\begin{example}
Suppose $S$ is a finite subset of $(\mathbb{Z}, \rho_2)$, $S=\{0,2,8,3\}$. Then a $\rho-$ordering of $S$ starts (arbitrarily) with $a_0=0$ , which forces $a_1=3$, since $\rho(0,3)=1=diam(S)$. The sequence continues $a_2=2$ and $a_3=8$, but after this point the sequence becomes arbitrary because  $\prod_{i=0}^{n-1} \rho(s,a_i)$ will contain a $0$, given by the repeated term. Indeed, for any finite subset $S$ with $\lvert S \rvert = n$, the $\rho-$ordering of $S$ is arbitrary from the $n^{th}$ point on. 
\end{example}

\begin{definition}
	\cite{kj} Let $\sigma(i)$ be a $\rho-$ordering of $S$. The \textbf{$\rho$-sequence} of $S$ is defined by letting $a_0=1$  and $a_n=\prod_{i=0}^{n-1} \rho(\sigma(n),\sigma(i))$, for $n > 0$.
\end{definition}

\begin{itemize}
\item this gives back the n-th diameter - the limit is the transfinite diameter and we take it as the definition of capacity
\item clean up all the sequence definitions to use the same format
\item check the historical background and this section to make sure you've said enough about fekete n-tuples and p-orderings corresponding
\item go through the JLC chabert reference
\item check that youve used the same notation for closed balls
\end{itemize}


\begin{proposition}
	\cite{kj} The $\rho$-sequence of $S$ is well-defined so long as $S$ is compact and $\rho$ is an ultrametric. That is, the $\rho$-sequence of a compact subset of an ultrametric spaces does not depend on the choice of $\rho$-ordering of $S$.
\end{proposition}

\begin{definition}
	\cite{kj}  Let $\gamma(n)$ be the $\rho$-sequence of $S$. The \textbf{valuative capacity} of $S$ is \[\omega(S)
	:= \lim_{n\to\infty} \gamma(n)^{1/n}\]  
\end{definition}


\begin{proposition}
	\cite{kj} For $S$ and $\gamma(n)$ as above,  $\lim_{n\to\infty} \gamma(n)^{1/n} = r < \infty$. 
\end{proposition}


\begin{proposition}
	If $S \subseteq M$ is a finite subset of an ultrametric space, then $\omega(S) =0$.
\end{proposition}


\begin{proposition}
	(upper bound) If $diam(S) = d$, then $\omega(S) < d$.
\end{proposition}

\begin{proof}
	Since $d$ is the diameter of $S$, the $n^{th}$ term of the $\rho$-sequence of $S$ is bounded by $d^n$ and so $ \lim_{n\to\infty} \gamma(n)^{1/n}=d$ if, and only if, $\gamma(n)=d^n$, $\forall n$. This implies $\rho(a_n, a_i) = d$, $\forall n$ and $\forall i < n$, but then $\rho(a_i,a_j)=d$, $\forall i,j$, since the $\rho$-sequence is maximized at each $n$. This means $\omega(S) < d$ would imply that the cover of $S$, $\cup_{a_i} B_d(a_i)$ is in fact an infinite partition, contradicting the compactness of $S$. Then  $\omega(S)= \lim_{n\to\infty} \gamma(n)^{1/n}<d$. 
\end{proof}

This doesn't work because  $\cup_{a_i} B_d(a_i)$  could fail to be a cover - the $\rho-$ ordering will be a dense subset under mild conditions - then being compact should imply a contradiction since we should have $\overline{S}=S$ for $S$ compact.

Make a comment about algebraic structure here

\begin{proposition}
\label{translation invariance}
	(translation invariance) If $(M, \rho)$ be a compact ultrametric space and s also a topological group for which $\rho$ is (left) invariant under the group operation, then $\omega$ is also (left)-invariant. That\ is, if $\rho(x,y)=\rho(gx,gy)$, $ \forall g,x,y \in M$, then $\omega(gS)=\omega(S)$, for $S \subseteq M$.	
\end{proposition}

\begin{proof}
	Let $\{a_i\}_{i=0}^\infty$ be a $\rho$-ordering for $S$. Then $\{ga_i\}_{i=0}^\infty$ is a $\rho$-ordering for $gS$. Then $$\omega(gS) = \lim_{n\to\infty} \gamma(n)^{1/n} =  \lim_{n\to\infty} [\prod_{i=0}^{n-1} \rho(ga_n,ga_i)]^{1/n} = \lim_{n\to\infty} [\prod_{i=0}^{n-1} \rho(a_n,a_i)]^{1/n}	 = \omega(S)$$
\end{proof}	

\begin{example}
	With the notation of the previous section, note that $\rho_p$ is translation invariant since for $x,y \in (\mathbb{Z}_p, \rho_p)$, $\rho_p(x,y) = p^{-v_p(x-y)} = p^{-v_p((a+x)-(a+y))} = \rho_p(a+x,a+y)$. Then $\omega(a+S) = \omega(S)$ for $S \subseteq (\mathbb{Z}_p, \rho_p)$.
\end{example}

\begin{proposition}
(scaling)	Let $(V, N)$ be a normed vector space and suppose $N$ satisfies the strong triangle identity, so that $(V,\rho_N)$ is an ultrametric space. Then if $N$ is multiplicative, so is $\omega$. That is, if $N(gx)=N(g)N(x)$,$\forall g,x \in V$, then $\omega(gS) = N(g)  \omega(S)$, for $g \in V$ and $S \subseteq M$. 
\end{proposition}

\begin{proof}
	Let $\rho$ be the metric induced by $N$, so that $\rho(x,y) = N(x-y), \forall x,y \in V$. Let $\{a_i\}_{i=0}^\infty$ be a $\rho$-ordering for $S$. Then since $N$ is multiplicative, for $u, v \in gS$, $u=gs_i$ and $v=gs_j$ for some $s_i, s_j \in S$,  $$\rho(u, v) = \rho(gs_i, gs_j) =N(gs_i - gs_j) = N(g(s_i - s_j)) = N(g)N(s_i - s_j) = N(g)\rho(s_i,s_j).$$
	Then $\{ga_i\}_{i=0}^\infty$ is a $\rho$-ordering for $gS$ and 
	
	$$\omega(gS) = \lim_{n\to\infty} [\prod_{i=0}^{n-1} \rho(ga_n,ga_i)]^{1/n} 
	= \lim_{n\to\infty} [\prod_{i=0}^{n-1} N(g)\rho(a_n,a_i)]^{1/n} $$
	$$= \lim_{n\to\infty} [N(g)^n\prod_{i=0}^{n-1} \rho(a_n,a_i)]^{1/n} = N(g) \lim_{n\to\infty} [\prod_{i=0}^{n-1} \rho(a_n,a_i)]^{1/n} = N(g) \omega(S)$$
\end{proof}


\begin{example}
	Since $\rho_p$ is multiplicative, $\omega(mS) = v_p(m)\cdot \omega(S)$ for $m \in \mathbb{Z}_p$ and $S \subseteq \mathbb{Z}$. In particular, $\omega(p\mathbb{Z}) = \frac{1}{p}\cdot \omega(\mathbb{Z})$. %= \frac{1}{p}\cdot p^{\frac{1}{1-p}} = p^{-p/p-1}.$
\end{example}

This is important and you should say so. The following proposition is from \cite{kj}, where it is given for some $S$  written as the union of two subsets, although it is easily seen to be true for $S$ equal to any finite union, so long as the other assumptions remain satisfied.

\begin{proposition}
	\cite{kj}(subadditivity) If  $diam(S) := \max_{x,y \in S} \rho(x,y)=d$ and $S=\cup_i^n A_i$ for $A_i$ compact subsets of $M$ with $\rho(A_i, A_j)=d, \forall i,j$, then \[\frac{1}{log(\omega(S)/d) } = \sum_{i=1}^n \frac{1}{log(\omega(A_i)/d)}\] 
\end{proposition}


\begin{example}
$\omega(\mathbb{Z})$
\end{example}

\begin{corollary}
	Suppose $S = \cup_i^n S_i$ with $\rho(S_i, S_j)=d=diam(S)$ and also $\omega(S_i)=\omega(S_j)$, $\forall i,j$ .  Let $r \in \mathbb{R}$ be such that $\omega(S_i)=r\omega(S)$, $\forall i$. Then $\omega(S) = r^{\frac{1}{n-1}}\cdot d$. In particular if $S = \mathbb{Z}$ and $(M,\rho)= (\mathbb{Z}, \mid \cdot\mid_p)$ then $\omega(S)=(\frac{1}{p})^{1/p-1}$ for any prime $p$. 
\end{corollary}

\begin{example}
\end{example}

\begin{corollary}
	(Joins of computable sets are computable) Let  $\Gamma_M = \{\gamma_0, \gamma_1,\ldots, \gamma_\infty=0\}$ be the set of distances in $M$. Suppose that $S = B_{\gamma_i}(x)$,  for some $x$ and $i$, is the union of $2$ or more balls of radius $\gamma_{i+1}$, i.e., $S=\cup_{j=1}^n B_{\gamma_{i+1}} (x_j)$ is a join in the lattice of open sets in $M$, then 
	\[\frac{1}{log(\omega(S)/\gamma_{i+1} )} = \sum_{j=1}^n \frac{1}{log(\omega(B_{\gamma_{i+1}}(x_j))/\gamma_{i+1} )}\]
\end{corollary}

