\section*{Background from Gerritzen and van der Put}

Background results from \cite{gvdp}. Let $k$ be a field that is complete with respect to a non-archimedean valuation and let $K$ be a complete and algebraically closed field containing $k$. 
\begin{definition*}\cite{gvdp}
	The set $\{\lambda \in k; \mid \lambda\mid \leq 1\}$, denoted $k^0$, is the \textbf{valuation ring} of $k$. It has a unique maximal ideal, denoted $k^{00}$, given by $\{\lambda \in k; \mid \lambda\mid < 1\}$. The \textbf{residue field} of $k$ is $\bar{k} := k^0/k^{00}$.
\end{definition*}


\begin{definition*}\cite{gvdp} The \textbf{projective line over $k$}, denoted $\mathbb{P}^1(k)$, is the space whose points are lines $l$ in $k^2$ that intersect $(0,0)$ and whose topology and field structure are inherited from $k$. 
\end{definition*}

We give two equivalent representations for the points of $\mathbb{P}^1(k)$. A point $p \in \mathbb{P}^1(k)$ is an equivalence class of $k^2 \setminus (0,0)$ under the relation $(x,y) \sim (x',y')$ if there exists a $\lambda \in k\setminus 0$ such that  $(x,y) = \lambda(x',y')$. Equivalently, suppose that $l$ is a line in $k^2$ intersecting the origin, that is a point in $\mathbb{P}^1(k)$. We denote $l$ by a representative $[x_0, x_1] \in k^2$ such that $l = \{\lambda (x_0, x_1 ); \lambda \in k\}$, called homogeneous coordinates of $l$.



\begin{proposition*} \cite{gvdp} Let $\psi: k \rightarrow \mathbb{P}^1(k)$ be the map given by $\psi(\lambda_0) = [1, \lambda_0]$, where $ [1, \lambda_0]$ is the line in $k^2$, $\{\lambda(1, \lambda_0); \lambda \in k\}$. Then the image of $\psi$ is $\mathbb{P}^1(k) \setminus [0,1]$ and is isomorphic to $k$, so that $k$ is identified with projective space minus a distinguished point, $[0,1]$, which is denoted by $\infty$.  
\end{proposition*}

\begin{definition*}\cite{gvdp} $k$ is called a \textbf{local field} if $k$ is locally compact. \end{definition*}

\begin{proposition*} \cite{gvdp} The following are equivalent:
	\begin{enumerate}
		\item $k$ is a local field.
		\item $\mid k^*\mid \cong \mathbb{Z}$ and $\bar{k}$ is finite, where $k*$ is the set of units in $k$, ie $k^* = \{\lambda \in k, \lambda \neq 0\}$.
		\item $k$ is a finite extension of either $\mathbb{Q}_p$ or $\mathbb{F}_p((t))$.
		\item $\mathbb{P}^1(k)$ is compact
	\end{enumerate}
\end{proposition*}	

\begin{definition*} \cite{gvdp}
	We denote by $GL(2,k)$ the set of invertible $2 \times 2$ matrices over $k$. A \textbf{fractional linear automorphism}, $\phi$, of $\mathbb{P}^1(k)$ is a map  defined by $z \mapsto \frac{az +b}{cz +d}$ for some 
	$\bigl( \begin{smallmatrix}a & b\\ c &d\end{smallmatrix}\bigr) \in GL(2,k)$. The set of fractional linear automorphisms of $\mathbb{P}^1(k)$ is denoted $PGL(2,k)$. Note that $PGL(2,k) = GL(2,k) / \{ \bigl( \begin{smallmatrix}\lambda & 0\\ 0 &\lambda \end{smallmatrix}\bigr); \lambda \in k^*  \}$. In homogeneous coordinates, we can represent the action of $\phi$ by $[x_0,x_1] \mapsto [cx_1 +dx_0, ax_1 +bx_0]$. 
\end{definition*}



\begin{definition*} \cite{gvdp}
	Suppose $\Gamma$ is a subgroup of $PGL(2,k)$. A point $p  \in \mathbb{P}^1(k)$ is a \textbf{limit point of $\Gamma$}, if there exists a point $q$ in  $\mathbb{P}^1(k)$ and a sequence $\{\gamma_n\}_{n\geq 1}$ in $\Gamma$ such that $\lim_{n\to\infty} \gamma_n(q) = p$.
\end{definition*}

\begin{proposition*} \cite{gvdp} If $\Gamma$ is not a discrete subgroup of $PGL(2,k)$ then every point of  $\mathbb{P}^1(k)$ is a limit point of $\Gamma$.
\end{proposition*}	

\begin{proof} 
	Since $\Gamma$ is not discrete, the sequence  $\{\gamma_n\}_{n\geq 1}$ has a limit $\gamma$ in $\Gamma$. Let $p$ be any point of $\mathbb{P}^1(k)$ and let $q= \gamma^{-1}(p)$. Then $\lim_{n\to\infty} \gamma_n(q) = \lim_{n\to\infty} \gamma_n(\gamma^{-1}(p)) =p$. 
\end{proof}	

\begin{definition*} \cite{gvdp} A subgroup $\Gamma$ of $PGL(2,k)$ is \textbf{discontinuous} if the closure of every orbit of $\Gamma$ in $\mathbb{P}^1(k)$ is compact and the set of all limit points of $\Gamma$ is not equal to  $\mathbb{P}^1(k)$ .
\end{definition*}

\begin{proposition*} \cite{gvdp} If $\Gamma$ is a discontinuous subgroup of $PGL(2,k)$  and $\mathcal{L}$ is the set of limit points of $\Gamma$, then $\mathcal{L}$ is compact, no where dense and if $\mathcal{L}$ contains more than two points, $\mathcal{L}$ is perfect.
\end{proposition*}

\begin{definition*} \cite{gvdp}
	Let $A$ be an element of $GL(2,k)$ and let $a_1$ and $a_2$  be the eigenvalues of $A$. Then $A$ is called \textbf{elliptic} if $a_1 \neq a_2$, but $\mid a_1\mid = \mid a_2\mid$. $A$ is called \textbf{parabolic} if $a_1 = a_2$, and $A$ is called \textbf{hyperbolic} if  $\mid a_1\mid \neq \mid a_2\mid$.
\end{definition*}

\begin{example} 
	Consider the matrix $T_s =  \bigl( \begin{smallmatrix}p & s\\ 0 & 1 \end{smallmatrix}\bigr) \in GL(2, \mathbb{Q}_p)$  for some $s$ in $(0,\ldots, p-1)$ (note that $det(T_s) = p$ is invertible in $\mathbb{Q}_p$, so that $T_s \in GL(2,\mathbb{Q}_p)$, although it is not in $GL(2, \mathbb{Z}_p)$). $T_s$ has eigenvalues $p$ and $1$ and so $T_s$ is hyperbolic for any choice of $s$ or $p$. Consider the action of $T_s$ on $\mathbb{Z}_p \subset \mathbb{Q}_p$, where $\mathbb{Z}_p$ is identified with the subspace $\{[1,\lambda];\lambda \in \mathbb{Z}_p \}$ of $\mathbb{P}^1(\mathbb{Q}_p)$. In homogeneous coordinates, this action is given by $[1,\lambda] \mapsto [1, p\lambda +s]$. Since $\mid (p\lambda +s -s)\mid = \mid p\lambda\mid \leq \frac{1}{p}$, $T_s$ sends $\lambda$ to $B_\frac{1}{p}(s)$. Also note that for $s = 0$, $T_s$ has the effect of shifting the index of $\lambda$ by 1, that is, if $\lambda = \sum_{i=n}^\infty a_ip^i$, where $n = ord(\lambda)$, then $T_0([1,\lambda]) = [1, p\lambda] \rightsquigarrow p\lambda = \sum_{i=n+1}^\infty a_{(i-1)}p^i$.	
\end{example}	


\section*{Computation of the capacity of some sets\\ (F\&P, section 5)}
\subsection*{Setup}
Let $A=\{0,1,..,d-1\}$ be a finite alphabet and $A^{\mathbb{N}}$ be the collection of infinite sequenes with values in $A$. Note $A^{\mathbb{N}}$ is a Cantor set, so it is perfect, nowhere dense, and compact.  \\

A basis for the topology is given by the cylinder set: take countably many copies of $\{0,1,...,d-1\}$ where each copy has the discrete topology.\\\\ Let $p \geq d$ be a prime number and let $\phi$ be the canonical embedding of $A^\mathbb{N}$ into $\mathbb{Z}_p$ via the following continuous (under the above topology) map: \[ \phi: A^{\mathbb{N}} \rightarrow \mathbb{Z}_p \text{ by } (x_n)_{n\geq0} \mapsto \sum_{k=0}^\infty x_kp^k\]

\begin{lemma*} (F\&P Lemma 5.1)\\ Let $w_1,w_2,\ldots,w_s$ be $s\geq 2$ words with the same length $l$ such that all the first letters are distinct. Let $X \subset A^{\mathbb{N}}$ be the set of sequences such that any factor is a factor of a concatenation of the words $w_1,w_2,\ldots,w_s$. Then the set $E := \phi(X) \subset \mathbb{Z}_p$ satisfies: \[E=\cup_{i=1}^s x_i +p^l E \text{,   with } x_i=\phi(w_i0^\infty)\]
It is a regular compact set and its valuative capacity is \[L_p(E) = \frac{l}{s-1}\] Notice that this provides examples of sets with empty interiors but with positive capacities.
\end{lemma*}
\begin{example}
	$w_1=0, w_2=2,  A=\{0,1,2\}, p=d=3$\\
	Then $\{x_n\}_{n\geq0} \in X$ if each term in  $\{x_n\}_{n\geq0}$ is either $0$ or $2$. We have \[E=0 + 3E \cup 2 + 3E \text{ and }\]  \[L_p(E) = \frac{1}{2-1} =1\] 
\end{example}

Note that we can rephrase the lemma as follows:\\ Let $x_1,x_2,\ldots,x_s$ be $s \geq 2$ points in $\mathbb{Z}_p$ such that $\mid x_i - x_j \mid_p = 1$, $\forall i,j \in 1,...,s$. Suppose also that there exists an $l \in \mathbb{N}$ such that  $\forall i$, $$x_i = \sum_{i=0}^{\infty} a_ip^i = \sum_{i=0}^{l} a_ip^i $$ Let $\gamma_i$ be the fractional linear automorphism of $\mathbb{P}^{1}(\mathbb{Q}_p)$ given by $\bigl( \begin{smallmatrix}p^l & x_i\\ 0 & 1 \end{smallmatrix}\bigr)$ and let $\Gamma$ be the subgroup of $PGL(2,\mathbb{Q}_p)$ generated by the $\gamma_i$. \\

Then $\Gamma$ has a subgroup $H$ such that the limit set $\mathcal{L}$ of $H$ has the property that $ Z = \psi^{-1}(\mathcal{L})$ is equal to $\phi(X)$ in the original lemma. In particular $Z$ is a regular, compact subset of $\mathbb{Z}_p$ satisfying

$$ Z= \cup_{i=1}^s x_i + p^lZ =\cup_{i=1}^s B_{\frac{1}{p^l}}(x_i)$$

and with vaulative capacity \[L_p(Z) = \frac{l}{s-1}\]


\begin{proof}
	We must show that the set $Z$ above is equal to $E=\phi(X)$ in the original lemma. First note that if $w_1, w_2,...,w_s$ are words in $A^\mathbb{N}$, then the first letter of each $w_i$ is distinct if and only if $\mid \phi(w_i) - \phi(w_j) \mid_p = 1, \forall i,j$  (since the pairwise distance is 1 if and only if  $ord(\phi(w_i) - \phi(w_j)) = 0$ for any $i$ and $j$, if and only if the coefficient of $p^0$ (i.e., the first letter each $w_i$) is different $\forall i,j$). So then the $x_i$ are just the $\phi(w_i)$.  \\
	
Now consider the limit set of $\Gamma$. Let $\gamma \in \Gamma$.
If $\gamma$ is a product of the generators $\gamma_i$, then $\gamma$ is represented by a matrix of the form:  $\bigl( \begin{smallmatrix}p^{lm} & z_m\\ 0 & 1 \end{smallmatrix}\bigr)$,  where $m \in \mathbb{N}$ and $z_m$ is an element of $\mathbb{Z}_p$ whose coefficient vector is a concatenation of the coefficient vectors of the $x_i$ (for $0 \leq i \leq ml$ and $0$ for $i > ml$). For example, $\gamma_i \gamma_j \gamma_k =$
 
\[\bigl( \begin{smallmatrix}p^{l} & x_i\\ 0 & 1 \end{smallmatrix}\bigr) \bigl( \begin{smallmatrix}p^{l} & x_j\\ 0 & 1 \end{smallmatrix}\bigr) \bigl( \begin{smallmatrix}p^{l} & x_k\\ 0 & 1 \end{smallmatrix}\bigr) = 
\bigl( \begin{smallmatrix}p^{3l} & p^{2l}x_k + p^lx_j +x_i\\ 0 & 1 \end{smallmatrix}\bigr)\]
\item The action of this map is given by \[[a_0,a_1] \mapsto [a_0, p^{lm}a_1 + za_0] \sim [1, p^{lm}\frac{a_1}{a_0} + z_m]\]
As $m$ tends to infinity, this point tends to $ [1, 0 + z]$, where $z$ is an element of $\mathbb{Z}_p$ whose entire coefficient vector is a concatenation of the $x_i$'s. The set $\psi^{-1}([1,z])$ for all such $z$ is exactly the set $E=\phi(X)$.\\

Now suppose $\gamma$ is a product of the inverses of the generators $\gamma_i^{-1}$, then $\gamma$ is represented by a matrix of the form:  $\bigl( \begin{smallmatrix}p^{-lm} & -p^{-l}z^{-1}\\ 0 & 1 \end{smallmatrix}\bigr)$,  where $m \in \mathbb{N}$ and $z$ is as above. For example, 
\[\bigl( \begin{smallmatrix}p^{-l} & -p^{-l}x_i\\ 0 & 1 \end{smallmatrix}\bigr) \bigl( \begin{smallmatrix}p^{-l} & -p^{-l}x_j\\ 0 & 1 \end{smallmatrix}\bigr) \bigl( \begin{smallmatrix}p^{-l} & -p^{-l}x_k\\ 0 & 1 \end{smallmatrix}\bigr) = 
\bigl( \begin{smallmatrix}p^{-3l} & -p^{-3l}x_k - p^{-2l}x_j - p ^{-l}x_i\\ 0 & 1 \end{smallmatrix}\bigr)\]
The action of this map is given by \[[a_0,a_1] \mapsto [a_0, p^{-lm}a_1  -p^{-lm}za_0] \sim [1, p^{-lm}(\frac{a_1}{a_0}- z)]\]
This time as $m$ grows, the image approaches infinity and so has empty preimage under $\psi$.\\

Lastly, we consider elements of $\Gamma$ made up of both generators and the inverse of generators. These elements will produce translations, either of the form  $\bigl( \begin{smallmatrix}1 & p^{-l}(x_i-x_j)\\\ 0 & 1 \end{smallmatrix}\bigr)$ if $\gamma=
 \gamma_j^{-1} \gamma_i$ or $\bigl( \begin{smallmatrix}1 & x_j-x_i\\\ 0 & 1 \end{smallmatrix}\bigr)$ if $\gamma =  \gamma_j \gamma_i^{-1}$.
These elements commute with each other and so the subgroup which they generate is normal. We can quotient by the entire translation subgroup, ie the subgroup generated by $\{\gamma_i \gamma_j^{-1}, \gamma_i ^{-1} \gamma_j; \forall i,j \in 1,\ldots,s\}$ to obtain $H$. Then $\mathcal{L} = \infty$  $\cup$
$\{[1,z] \mid z \in \mathbb{Z}_p \text{ and the coeffecient vector of } z  \text{ is a concatenation of the coeffiecient vectors of the }x_i\text{'s}\}$
and $\psi^{-1}(\mathcal{L})=E=\phi(X)$, as required.
%The resulting quotient group is discontinuous, finitely generated and every element ($\neq id$)  is hyperbolic, ie it is a Schottky group. 



\end{proof}
 
 
More background: 

\begin{definition*} \cite{gvdp} A \textbf{Schottky group} is a finitely-generated, free and discontinuous subgroup of  $PGL(2,k)$
\end{definition*}

%Observation: suppose $z \in \mathbb{Z}_p$ is such that $z = \sum_{i=0}^\infty a_ip^i = \sum_{i=0}^n a_ip^i$ for some $n \in \mathbb{N}$. Then $B_{\frac{1}{p^n}}(z)$ is the set of all $y \in \mathbb{Z}_p$ of the form $y=\sum_{i=0}^n a_ip^i + \sum_{i=n+1}^\infty b_ip^i$.

%Note that:\newline $0 + 3E = \{3z \in \mathbb{Z}_3; z = \sum_{i=0}^\infty a_ip^i, a_i \in {0,2}\} =\\
%\{z \in \mathbb{Z}_3; z = \sum_{i=1}^\infty a_ip^i, a_i \in {0,2}\} =\\
%\{z \in \mathbb{Z}_3; ord(z) \geq 1 \} \cap E=\\
%\{z \in \mathbb{Z}_3; \mid z \mid_3 \leq \frac{1}{3}\} \cap E = \\
%\{z \in \mathbb{Z}_3; \mid 0- z \mid_3 \leq \frac{1}{3}\} \cap E = B_{\frac{1}{3}}(0) \cap E$.
%\\\\and:\\ 
%$2 + 3E = \{2 + 3z \in \mathbb{Z}_3; z = \sum_{i=0}^\infty a_ip^i, a_i \in {0,2}\} =\\
%\{2 + z \in \mathbb{Z}_3; z = \sum_{i=1}^\infty a_ip^i, a_i \in {0,2}\} =\\
%\{2 + z \in \mathbb{Z}_3; ord(z) \geq 1 \} \cap E=\\
%\{z \in \mathbb{Z}_3; \mid 2 - z \mid_3 \leq \frac{1}{3}\} \cap E = \\
%B_{\frac{1}{3}}(2) \cap E$
%\\ and so $E = B_{\frac{1}{3}}(0)\cap E \cup B_{\frac{1}{3}}(2) \cap E$


