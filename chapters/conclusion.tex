Traditionally, the capacity of a metric space indicates the extent to which points, or charges, can spread out within the space. By leveraging number-theorectic concepts introduced by Bhargava and K. Johnson, we have explored some properties of capacity in novel spaces.  As part of this process, we have seen that some classical results carry over, and at the same time some new results, unique to our setting, have been introduced.\\

For example, we have seen that capacity in ultrametric spaces respects translation and scaling actions, as it does over $\mathbb{C}$, and submits to a decomposition formula. On the other hand, we have also seen that, in an ultrametric space, the way in which these spaces decompose interacts with the computability of capacity. In the final chapter, we examined spaces that lacked a field structure and which decomposed in an aperiodic way. In this case, this aperiodic decomposition meant that not only was a closed form for capacity eluded, but it also led us to doubt the algebracity of capacity in this setting.