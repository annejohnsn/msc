\documentclass{beamer}
\theoremstyle{definition}
\newtheorem{proposition}{Proposition}
\newtheorem*{proposition*}{Proposition}
\newtheorem{definition_proposition}{Definition-Proposition}
\newtheorem*{definition_proposition*}{Definition-Proposition}

\title{$\rho$-orderings and valuative capacity in ultrametric spaces}
\author{Anne Johnson}
\date{\today}



\begin{document}
\maketitle


\begin{frame}
\frametitle{Ultrametric basics}
\begin{definition}
A \textbf{$p-$ordering} of an infinite set, $S \subseteq \mathbb{Z}_p$, is a sequence in $S$ such that for $n > 0$, $a_n$ minimizes
\[v_p((x - a_{n-1}) \ldots (x - a_0))\] 
\end{definition}

\begin{definition_proposition}
 The \textbf{$p$-sequence} of $S$ is the sequence whose $0^{th}$-term is $1$ and whose $n^{th}$ term, for $n >0$, is $v_p ((a_n-a_{n-1}) \ldots (a_n - a_0))$.
\end{definition_proposition}

\end{frame}


\begin{frame}
\frametitle{$\rho$-orderings, $\rho$-sequences, and valuative capacity}
In what follows, let $S$ be a compact subset of an ultrametric space $(M,\rho)$.
\begin{definition}
\cite{kj} A \textbf{$\rho$-ordering} of $S$ is a sequence $\{a_i\}_{i=0}^\infty \subseteq S$ such that $\forall n > 0$, $a_n$ maximizes $\prod_{i=0}^{n-1} \rho(s,a_i)$ over $s \in S$. 
\end{definition}



\end{frame}

\begin{frame}
\frametitle{$\rho$-orderings, $\rho$-sequences, and valuative capacity}
\begin{definition_proposition}
\cite{kj}  Let $\gamma(n)$ be the $\rho$-sequence of $S$. The \textbf{valuative capacity} of $S$ is \[\omega(S)
 := \lim_{n\to\infty} \gamma(n)^{1/n}\]  
\end{definition_proposition}

\end{frame}


\begin{frame}
\frametitle{valuative capacity: quick results}
\begin{proposition*}
	(translation invariance) Let $(M, \rho)$ be a compact ultrametric space and suppose $M$ is also a topological group. If $\rho$ is (left) invariant under the group operation, then so is $\omega$. That\ is, if $\rho(x,y)=\rho(gx,gy)$, $ \forall g,x,y \in M$, then $\omega(gS)=\omega(S)$, for $S \subseteq M$.	
\end{proposition*}

\begin{proposition*}
(scaling) Let $(V, N)$ be a normed vector space and suppose $N$ satisfies the strong triangle identity. Then if $N$ is multiplicative, so is $\omega$. That is, if $N(gx)=N(g)N(x)$,$\forall g,x \in V$, then $\omega(gS) = N(g)  \omega(S)$, for $g \in V$ and $S \subseteq M$. 
\end{proposition*}
\end{frame}

\begin{frame}
\frametitle{valuative capacity: subadditivity}

\begin{proposition*}
\cite{kj}(subadditivity) If  $diam(S) := \max_{x,y \in S} \rho(x,y)=d$ and $S=\cup_i^n A_i$ for $A_i$ compact subsets of $M$ with $\rho(x_i, x_j)=d$, $\forall x_i \in A_i$, $\forall x_j \in A_j$ and $\forall i,j$, then \[\frac{1}{log(\omega(S)/d) } = \sum_{i=1}^n \frac{1}{log(\omega(A_i)/d)}\] 
\end{proposition*}

\begin{corollary}
Suppose $S = \cup_i^n S_i$ with $\rho(S_i, S_j)=d=diam(S)$ and also $\omega(S_i)=\omega(S_j)$, $\forall i,j$ .  Let $r \in \mathbb{R}$ be such that $\omega(S_i)=r\omega(S)$, $\forall i$. Then $\omega(S) = r^{\frac{1}{n-1}}\cdot d$. In particular if $S = \mathbb{Z}$ and $(M,\rho)= (\mathbb{Z}, \mid \cdot\mid_p)$ then $\omega(S)=(\frac{1}{p})^{1/p-1}$ for any prime $p$. 
\end{corollary}
\end{frame}



\begin{frame}
\frametitle{Constructing a $\rho$-ordering}
Setup:
\begin{itemize}
\item Let $S \subseteq M$ be a compact subset of an ultrametric space $(M, \rho)$. 
\item Let $\Gamma_S =\{\gamma_0, \gamma_1,\ldots,\gamma_\infty=0\}$ be the set of distances in $S$.  
\item Note that for each $k \in \mathbb{N}$, the closed balls of radius $\gamma_k$ partition $S$. That is, \[S=S_{\gamma_k} := \cup_{i=1}^n \overline{B_{\gamma_k}(x_i)}\] where both $n$ and the $x_i$'s depend on $k$.
\end{itemize}

\end{frame}

\begin{frame}
\frametitle{Constructing a $\rho$-ordering}
Setup, continuted:
\\Fix a $k \in \mathbb{N}$. 
\begin{itemize}
\item Let $S_{\gamma_k} = \cup_{i=1}^n \overline{B_{\gamma_k}(x_i)}$ be  a partition of $S$, as above. 
 \item Note that we can construct $S_{\gamma_{k+1}}$ by partitioning each of the $\overline{B_{\gamma_k}(x_i)}$ , i.e., \[S = S_{\gamma_{k+1}} = \cup_{i=1}^n \cup_{j=1}^{l_i} \overline{B_{\gamma_{k+1}}(x_{i,j})}\] where $1 \leq l_i \leq n$ and $\cup_{j=1}^{l_i} \overline{B_{\gamma_{k+1}}(x_{i,j})}=\overline{B_{\gamma_k}(x_i)}$, $\forall i$. 
\item We denote by $x_{i,j}$ the centre of a ball of radius $\gamma_{k+1}$ partitioning the ball $B_{\gamma_k}(x_i)$. \item Without loss of generality, when $j=1$, assume $x_{i,j}=x_i$, $\forall i$.\\
\end{itemize}

\end{frame}

\begin{frame}
\frametitle{Constructing a $\rho$-ordering}
We now make the following observation due to \cite{na},

\begin{lemma}
For each $k \in \mathbb{N}$, the elements of $S_{\gamma_k}$, that is, the closed balls of radius $\gamma_k$, themselves form an ultrametric space, where 
\[ \rho_k\overline{(B_{\gamma_k}(x)},\overline{B_{\gamma_k}(y)}) = 
\begin{cases}
\rho(x,y), & \text{if } \rho(x,y) > \gamma_k \\
0, & \text{if }   \rho(x,y) \leq \gamma_k \text{, i.e., } \overline{B_{\gamma_k}(x)}=\overline{B_{\gamma_k}(y)}
\end{cases}
\]
\end{lemma}
\end{frame}

\begin{frame}
\frametitle{Constructing a $\rho$-ordering}
We make the following observations:
\begin{itemize}
\item Since $S$ is  compact,  $S_{\gamma_k}$ is a finite metric space $\forall k < \infty$ and $S_{\gamma_\infty}=\cup_{x \in S}\overline{B_0(x)} = \cup_{x \in S}x=S$ and $\rho_\infty=\rho$. 
\item View $S_{\gamma_k}$, for fixed $k < \infty$ as a finite ultrametric space with $n$ elements. Let us denote an element of  $S_{\gamma_k}$, that is a $\overline{B_{\gamma_k}(x_i)}$, by its centre, $x_i$. 
\item Without loss of genearlity, we can reindex the $x_i$'s so that they give the first $n$ terms of a $\rho_k$-ordering of $S_{\gamma_k}$.
\end{itemize}
\end{frame}	


\begin{frame}
\frametitle{Constructing a $\rho$-ordering}
Setup, revisited:
\begin{itemize}
\item Let $S_{\gamma_k} =  \cup_{i=1}^n \overline{B_{\gamma_k}(x_i)}$ be the finite metric space describe above, and suppose the $x_i$ are indexed according to a $\rho_k$-ordering of  $S_{\gamma_k}$.
\item Let  $S_{\gamma_{k+1}}$ be the finite metric space formed by partitioning each of the ${B_{\gamma_k}(x_i)}$, so that  $S_{\gamma_{k+1}} = \cup_{i=1}^n \cup_{j=1}^{l_i} \overline{B_{\gamma_{k+1}}(x_{i,j})}$ and $x_{i,j}$ is a point in the ball $B_{\gamma_k}(x_i)$ with the convention that $x_{i,1} = x_i$, $\forall i$. 
\end{itemize}
\end{frame}

\begin{frame}
\frametitle{Constructing a $\rho$-ordering}
Consider the matrix $A_k$, whose $(i,j)^{th}$-entry is $x_{i,j}$ (or * if $l_i < j$).
\[A_k=
 \begin{pmatrix}
  x_{1,1} & x_{2,1} & \ldots  &x_{n,1} \\
  x_{1,2} & x_{2,2} &\ldots &x_{n,2} \\
  \vdots & \vdots & \ddots & \vdots \\
  x_{1,l_1} & x_{2,l_2} & \ldots &x_{n,l_n}
 \end{pmatrix}
\]
\pause
 A $\rho_{k+1}$-ordering of $S_{\gamma_{k+1}}$ can be found by concatenating the rows of $A_k$ (and ignoring *'s).
\end{frame}

\begin{frame}
\frametitle{Some corollaries}
\begin{corollary}
Interweaving the bottown row of the lattice of closed balls for a set $S$ gives a $\rho$-ordering of $S$. In particular, the natural ordering on the integers gives a $\rho_p$-ordering for every prime $p$. 

\end{corollary}

\begin{corollary}
Suppose $S$ and $T$ are compact subsets of an ultrametric space $M$ with $\Gamma_S = \Gamma_T$ and $\mid S_{\gamma_k}\mid =\mid T_{\gamma_k}\mid$, $\forall k$. Then $\omega(S) = \omega(T)$. 
\end{corollary}
\end{frame}



\begin{frame}
\frametitle{references}
\begin{thebibliography}{1}
\bibitem{ar} Alain M. Robert, A course in p-adic analysis.
\bibitem{kj} Keith Johnson, P-orderings and Fekete sets
\bibitem{na} Nate Ackerman,  Completeness in Generalized Ultrametric Spaces
\end{thebibliography}
\end{frame}

\end{document}