\documentclass{beamer}
\theoremstyle{definition}
\newtheorem{proposition}{Proposition}
\newtheorem*{proposition*}{Proposition}
\newtheorem{definition_proposition}{Definition-Proposition}
\newtheorem*{definition_proposition*}{Definition-Proposition}
\newtheorem*{lemma*}{Lemma}
\newtheorem*{definition*}{Definition}

\title{Valuative Capacity of of some compact subsets of $\mathbb{Z}_p$}
\author{Anne Johnson}
\date{\today}



\begin{document}
\maketitle

\begin{frame}
\frametitle{Background: $p$-orderings, $p$-sequences}
A \textbf{$p-$ordering} of an infinite set, $S \subseteq \mathbb{Z}_p$, is a sequence in $S$ such that for $\forall n > 0$, $a_n$ minimizes
\[v_p((x - a_{n-1}) \ldots (x - a_0))\] 
\pause \newline
cf: A \textbf{$\rho$-ordering} of $S$, a (compact) subset of an ultrametric space $(M, \rho)$,  is a sequence in $S$ such that $\forall n > 0$, $a_n$ maximizes 
\[\prod_{i=0}^{n-1} \rho(x,a_i)\] 
\end{frame}

\begin{frame}
\frametitle{Background: $p$-orderings, $p$-sequences}

 The \textbf{$p$-sequence} of $S$ is the sequence whose $0^{th}$-term is $1$ and whose $n^{th}$ term, for $n >0$, is 
\[v_p ((a_n-a_{n-1}) \ldots (a_n - a_0))\]
\pause \newline
cf:  The \textbf{$\rho$-sequence} of $S$ is the sequence whose $0^{th}$-term is $1$ and whose $n^{th}$ term, for $n >0$, is
\[\prod_{i=0}^{n-1} \rho(a_n,a_i)\]

\end{frame}


\begin{frame}
\frametitle{Background: valuative and logarithm capacity}
 The \textbf{valuative capacity} of an infinite set, $S \subseteq \mathbb{Z}_p$, is
\[ L_p(S) := \lim_{n\to\infty}  \frac{w_S(n,p)}{n}\]
where $w_S(n,p)$ is the $p-$sequence of $S$. 
\newline
\\nb: this is the Robin's constant and can be found via the equilibrium measure:
 \[L_p(S) = \inf_{\mu\in\mathcal{P}(\bar{S}) } \int \int v_p(x-y)d\mu(x)d\mu(y)\]
\end{frame}

\begin{frame}
\frametitle{Background: valuative and logarithm capacity}
 The \textbf{logarithm capacity} of an infinite set, $S \subseteq \mathbb{Z}_p$, is
\[ V_p(E) := p^{-L_p(E)}\]
\newline
nb: this is equal to the transfinite diameter and the Chebychev constant.
\pause \newline
%\\cf: The \textbf{generalized valuative capacity} of a set $S$ is \[\lim_{n\to\infty} \gamma(n)^{1/n}\] where $\gamma(n)$ is the $\rho-$sequence of $S$. \\nb: capacity is translation invariant and multiplicative (when the metric is) and has a subadditivity property.
\end{frame}

\begin{frame}
\frametitle{Fare and Petite, Lemma 5.1}
Let $A=\{0,1,..,d-1\}$ be a finite alphabet and $A^{\mathbb{N}}$ be the collection of infinite sequenes with values in $A$. 
\parskip=20pt

Let $p \geq d$ be a prime number and let $\phi$ be the canonical embedding of $A^\mathbb{N}$ into $\mathbb{Z}_p$ via the following continuous  map: \[ \phi: A^{\mathbb{N}} \rightarrow \mathbb{Z}_p \text{ by } (x_n)_{n\geq0} \mapsto \sum_{k=0}^\infty x_kp^k\]
\end{frame}

\begin{frame}
\frametitle{Fare and Petite, Lemma 5.1}
\begin{lemma*} \textit{Let $w_1,w_2,\ldots,w_s$ be $s\geq 2$ words with the same length $l$ such that all the first letters are distinct. Let $X \subset A^{\mathbb{N}}$ be the set of sequences such that any factor is a factor of a concatenation of the words $w_1,w_2,\ldots,w_s$. Then the set $E := \phi(X) \subset \mathbb{Z}_p$ satisfies: \[E=\cup_{i=1}^s x_i +p^l E \text{,   with } x_i=\phi(w_i0^\infty)\]
It is a regular compact set and its valuative capacity is \[L_p(E) = \frac{l}{s-1}\] Notice that this provides examples of sets with empty interiors but with positive capacities.}
\end{lemma*}
\end{frame}

\begin{frame}
\frametitle{Fares and Petite, Lemma 5.1}
An example:\parskip=20pt


	$w_1=0, w_2=2,  A=\{0,1,2\}, p=d=3$\\
	Then $\{x_n\}_{n\geq0} \in X$ if each term in  $\{x_n\}_{n\geq0}$ is either $0$ or $2$. We have \[E=0 + 3E \cup 2 + 3E \text{ and }\]  \[L_p(E) = \frac{1}{2-1} =1\] 

\end{frame}

\begin{frame}
\frametitle{Digression: projective $k$-space}
 Let $k$ be a field that is complete with respect to a non-archimedean valuation.

\begin{definition*} The \textbf{projective line over $k$}, denoted $\mathbb{P}^1(k)$, is the space whose points are lines $l$ in $k^2$ that intersect $(0,0)$.% and whose field structure is inherited from $k$. 
\end{definition*}


\begin{proposition*} Let $\psi: k \rightarrow \mathbb{P}^1(k)$ be the map given by $\psi(\lambda_0) = [1, \lambda_0]$, where $ [1, \lambda_0]$ is the line in $k^2$, $\{\lambda(1, \lambda_0); \lambda \in k^*\}$. Then the image of $\psi$ is $\mathbb{P}^1(k) \setminus [0,1]$ and is isomorphic to $k$, so that $k$ is identified with projective space minus a distinguished point, $[0,1]$, which is denoted by $\infty$.  
\end{proposition*}
\end{frame}


\begin{frame}
\frametitle{Digression: projective $k$-space}
\begin{definition*}
	We denote by $GL(2,k)$ the set of invertible $2 \times 2$ matrices over $k$. A \textbf{fractional linear automorphism}, $\phi$, of $\mathbb{P}^1(k)$ is a map  defined by $z \mapsto \frac{az +b}{cz +d}$ for some 
	$\bigl( \begin{smallmatrix}a & b\\ c &d\end{smallmatrix}\bigr) \in GL(2,k)$. The set of fractional linear automorphisms of $\mathbb{P}^1(k)$ is denoted $PGL(2,k)$.
\parskip=20pt

 Note that $PGL(2,k) = GL(2,k) / \{ \bigl( \begin{smallmatrix}\lambda & 0\\ 0 &\lambda \end{smallmatrix}\bigr); \lambda \in k^*  \}$. In homogeneous coordinates, we can represent the action of $\phi$ by $[x_0,x_1] \mapsto [cx_1 +dx_0, ax_1 +bx_0]$. 
\end{definition*}
\end{frame}

\begin{frame}
\frametitle{Digression: projective $k$-space}
\begin{definition*}
	Suppose $\Gamma$ is a subgroup of $PGL(2,k)$. A point $p  \in \mathbb{P}^1(k)$ is a \textbf{limit point of $\Gamma$}, if there exists a point $q$ in  $\mathbb{P}^1(k)$ and a sequence $\{\gamma_n\}_{n\geq 1}$ in $\Gamma$ such that $\lim_{n\to\infty} \gamma_n(q) = p$.
\end{definition*}

\end{frame}

\begin{frame}
\frametitle{Fares and Petite, Lemma 5.1, repharsed (1/2)}
 Let $x_1,x_2,\ldots,x_s$ be $s \geq 2$ points in $\mathbb{Z}_p$ such that $\mid x_i - x_j \mid_p = 1$, $\forall i,j \in 1,...,s$. Suppose also that there exists an $l \in \mathbb{N}$ such that  $\forall i$, $$x_i = \sum_{i=0}^{\infty} a_ip^i = \sum_{i=0}^{l} a_ip^i $$
\end{frame}

\begin{frame}
\frametitle{Fares and Petite, Lemma 5.1, repharsed (2/2)}
 Let $\gamma_i$ be the fractional linear automorphism of $\mathbb{P}^{1}(\mathbb{Q}_p)$ given by $\bigl( \begin{smallmatrix}p^l & x_i\\ 0 & 1 \end{smallmatrix}\bigr)$ and let $\Gamma$ be the subgroup of $PGL(2,\mathbb{Q}_p)$ generated by the $\gamma_i$.\\
If $\mathcal{L}$ is the limit set of $\Gamma$, and $Z$ is the subset of $\mathbb{Q}_p$ such that $Z = \psi^{-1}(\mathcal{L})$, (where $\psi: \mathbb{Q}_p \rightarrow \mathbb{P}^1(\mathbb{Q}_p)$ is the map given by $\psi(\lambda_0) = [1, \lambda_0]$) then $Z$ is a regular, compact subset of $\mathbb{Z}_p$ satisfying

$$ Z= \cup_{i=1}^s x_i + p^lZ =\cup_{i=1}^s B_{\frac{1}{p^l}}(x_i)$$

and with vaulative capacity \[L_p(Z) = \frac{l}{s-1}\]

\end{frame}

\begin{frame}
\frametitle{Fares and Petite, Lemma 5.1, repharsed}
Sketch of proof:
\begin{itemize}
\item We have to show w that the set $Z$ above is equal to $E=\phi(X)$ in the original lemma.
\item That that $w_i$ correspond to the $x_i$ is not hard to see.
\end{itemize}
\end{frame}

\begin{frame}
\frametitle{Fares and Petite, Lemma 5.1, repharsed}
What is the limit set of $\Gamma$?
\begin{itemize}
\item An element of $\Gamma$ is of the form $\bigl( \begin{smallmatrix}p^{lm} & z\\ 0 & 1 \end{smallmatrix}\bigr)$, where $m \in \mathbb{N}$ and $z$ is an element of $\mathbb{Z}_p$ whose coefficient vector is a concatenation of the coefficient vectors of the $x_i$ (for $0 \leq i \leq ml$ and $0$ for $i > ml$)
\item  Let $a=[a_0,a_1] \in \mathbb{P}^1(\mathbb{Q}_p)$ and let $\{\gamma_n\}$ be a sequence in $\Gamma$. We have that $$\lim_{n\to\infty} \gamma_n(a) = \lim_{n\to\infty} [a_0, p^{nl}a_1 + z_n] = [a_0, z],$$ where the coefficient vector of each $z_n$ is a concatenation of the coefficient vectors of the $x_i$, for finitely-many terms (and then $0$s), and $z$ is an element of $\mathbb{Z}_p$ whose entire coefficient vector is a concatenation of the coefficient vectors of the $x_i$.

\end{itemize}

\end{frame}


\begin{frame}
\frametitle{references}
\begin{thebibliography}{1}
\bibitem{ar} Youssef Fares and Samuel Petite, The valuative capacity of subshifts of finite type.
\bibitem{kj} Keith Johnson, P-orderings and Fekete sets
\end{thebibliography}
\end{frame}

\end{document}